%% !TEX root = ../md2-user-handbook.tex
%% !TeX spellcheck = en_GB
%% !TeX program = xelatex


\todo{@ALL review pls}
Each target platform has its own generator class that needs to adhere to the interface \\ \lstinline|IPlatformGenerator|. Each generator class is registered in the \lstinline|MD2RuntimeModule| and gets injected into the framework automatically. During modelling the Xtext Builder participant starts the building process every time a model file is changed and regenerates the platform files.
Prior to generating the source code, the model gets transformed and is passed on to the implementing generator classes. Basically, this step can be considered a model-to-model transformation with the aim to simplify the generation process.

\subsection{Autogenerator}
The autogenerator is a feature that allows to easily create view elements from model definitions. During the preprocessing, the references to content providers are resolved and content elements are created based on the model attribute types that the content provider declare. The schema for generation is as follows:

\begin{itemize}
	\item \lstinline!IntegerType, FloatType, StringType! become text input fields
	\item \lstinline!DateType, TimeType, DateTimeType! become text input fields with corresponding time type attribution (later depicted as date/time pickers)
	\item \lstinline!BooleanType! becomes a checkbox field
	\item \lstinline!ReferencedType!
		\subitem \lstinline!Enum! becomes an \enquote{option input} field
		\subitem \lstinline!Entites! are processed recursively and all elements are wrapped in a \enquote{flow layout pane}
\end{itemize}

\subsubsection{Remarks}
With each content element, a new \lstinline!MappingTask! is created that maps the content element to the attribute provided by the content provider. For this purpose a \enquote{custom action} called autoGenerationAction is created that the platform generators need to parse manually. If the view element is unmapped on startup or a user-specified mapping is found, the auto-generated mapping is removed.

In case the model attribute, from which a content element is created, has the optional parameters \lstinline!name! or \lstinline!description! set, these values are converted to label and tooltip representations for text inputs, option inputs and checkboxes. If no name is given by the modeller, the ID of the content element will be assigned as a label name with each uppercase letter preceded by a whitespace (\eg \enquote{myAddress} will be transformed into \enquote{My Address}).

\subsubsection{Cloning and references}
The \MD language allows  the modeller to define certain view elements once and then reuse them multiple times. Internally, these elements are copied and references pointing to them are resolved during the preprocessing. The same behavior applies to view elements that are referenced in a view container (\eg the flow layout pane) but are defined outside of this container. The following example shall serve to illustrate these two use cases.

\begin{lstlisting}[language=MD2]
TabbedPane mainView {
	customerPane -> customerView(tabTitle "Customer")
	generalPane -> generalView(tabTitle "General")
}
FlowLayoutPane customerPane(vertical) {
	headerPane
	TextInput myTextInput
}
FlowLayoutPane generalPane(vertical) {
	headerPane
	// Some view elements
}
FlowLayoutPane headerPane {
	Image logoImage("./capitol.png")
}
\end{lstlisting}

The code excerpt shows a tabbed pane that references two container elements. These containers again reference the same sub-container twice. After preprocessing the model appears to the code generators as follows:

\begin{lstlisting}[language=MD2]
TabbedPane mainView {
	FlowLayoutPane customerView(tabTitle "Customer", vertical) {
		FlowLayoutPane headerPane {
			Image logoImage("./capitol.png")
	    }
		TextInput myTextInput
	}
	FlowLayoutPane generalPane(tabTitle "General", vertical) {
		FlowLayoutPane headerPane {
			Image logoImage("./capitol.png")
		}
	}
}
\end{lstlisting}

\subsubsection{Resolving references}
Because view elements can be reused and referenced across the whole model, \MD needs to provide means on how to use these virtual elements in behavioral elements like actions. The problem is that Xtext only allows to reference the originating element, but not the reference itself. In the example provided this means that only \lstinline!headerPane! can be referenced, but any \lstinline!headerPane! in a \lstinline!customerPane! or \lstinline!generalPane! can not, because there is no such element during runtime. To solve this issue \MD has the language type \lstinline!AbstractViewGUIElementRef! that deals with these kind of references. This element allows to chain references to any element of the model in an arbitrary order. However, the scope has to be restricted to offer just the possible elements during autocompletion. During preprocessing the pseudo-referenced elements are converted into real elements and any reference to them will be resolved.

The reference to \lstinline!myTextInput! that resides in \lstinline!customerPane! and is part of the tabbed pane \lstinline!mainView! can be referenced as \lstinline!mainView.customerView->customerPane.myTextInput!. Basically, two elements are chained with \lstinline!->! as the delimiter.

The modeller also might want to reference an auto-generated content element. This is done by referencing the autogenerator followed by the model element in square brackets. In the following scenario the text input field that is generated for the customer’s first name shall be referenced.

\todo{check for bracket @ALL}
\begin{lstlisting}[language=MD2]
FlowLayoutPane myPane {
	AutoGenerator customerAutoGenerator {
		contentProvider customerContentProvider
		
		)
	}
}
contentProvider Customer customerContentProvider {
	providerType default
}

entity Customer {
	firstName: string
}
\end{lstlisting}
~
\\
The reference will then be \lstinline!myPane.customerAutoGenerator[Customer.firstName]!.

\subsubsection{Remarks}
When a view element, which is referenced or reused at a different place, is copied, each event binding, mapping or validator binding pointing to this view element will be copied as well. The copied version of the controller elements will be redirect to the copied version of the view elements. This does not work the other way around: any behavior element pointing to a referencing view element does not apply to the original element.

\subsection{Validators}
To ensure proper data integrity and to validate user input, each content element can be attributed with validators in a ValidatorBindingTask. When a view element is mapped to a model attribute some validators can be inferred automatically based on the attribute type (\eg for integers) and its parameters (\eg optional attributes). In these cases a validator is automatically created and bound to the view element in question. However, the modeler can still overwrite or unbind these validators in any actions that are performed upon startup.

\subsubsection{Type specific}
\begin{itemize}
\item \lstinline!IntegerType! enforces a StandardIsIntValidator being bound
\item \lstinline!FloatType! enforces a StandardIsNumberValidator being bound
\item \lstinline!StringType! enforces a StandardStringRangeValidator being bound
\end{itemize}

\subsubsection{Type Parameter specific}

\begin{itemize}
\item Omitting the \lstinline!optional! keyword will result in a StandardNotNullValidator being bound
\item Setting \lstinline!min! or \lstinline!max! values for any kind of attribute will result in an appropriate validator being bound that ensures the range for this data type
\end{itemize}

\subsection{Replacements}
\subsubsection{Enums}
The \MD language allows to declare enums explicitly and implicitly. Internally all enums will be treated as explicitly defined enums and converted accordingly. \\ \\
Before:
\begin{lstlisting}[language=MD2]
entity User {
	gender: {"male", "female"}
}
\end{lstlisting}
~
\\
After:
\begin{lstlisting}[language=MD2]
entity User {
	gender: User_Gender
}
enum User_Gender {
	"male", "female"
}
\end{lstlisting}


\subsubsection{Process Chains}
Nested process chains will be flattened. \\ \\
Before:
\begin{lstlisting}[language=MD2]
processChain pc1 {
	step firstStep:
		view mainView.customerView
	step nestedStep:
		subProcessChain pc2
	step thirdStep:
		view mainView.resultView
}
processChain pc2 {
	step secondStep:
		view mainView.InputView
}
\end{lstlisting}
~
\\
After:
\begin{lstlisting}[language=MD2]
processChain pc1 {
	step firstStep:
		view mainView.customerView
	step secondStep:
		view mainView.InputView
	step thirdStep:
		view mainView.resultView
}

\end{lstlisting}

\subsubsection{CombinedAction}
The sole use of combined actions is to trigger execution of other actions. This behavior can also be achieved by using call tasks in custom actions. Hence, all combined actions will be converted to custom actions internally.

\subsubsection{Miscellaneous}
Some minor adjustments are made to the model:

\begin{itemize}
\item Check for existence of the \lstinline!flowDirection! parameter for all flow layout panes.
\item Duplicate spacer according to the specified number of spacers.
\item Replace all named colours by their hex colour equivalents.
\item Replace custom validators with standard validator definitions.
\end{itemize}

\subsubsection{TestGenerator}
Usually model transformations are transparent to the modeller and even for the developer hard to trace. With the test generator, though, there is a way to get a glimpse of the model's state as XMI definition before it gets passed to the platform generators.
