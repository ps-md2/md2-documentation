% !TEX root = ../md2-user-handbook.tex
% !TeX spellcheck = en_GB
% !TeX program = xelatex

\lstset{language=Simple}

\label{sec:dev-mapapps}

The current implementation of the \MD framework generates web-based apps for a framework called 
\mapapps, which is mainly written in JavaScript. Code generation for Android and iOS applications are also targeted, but not fully implemented yet.

The generated code for \mapapps can be subdivided into three parts: static \mapapps code, dynamically generated \mapapps code and a backend. The static \mapapps code contains the part of the code which does not depend on the models created in the \MD DSL. Since it is static, it does not need to be generated, but is required for the overall functionality of the generated apps. The dynamically generated part is dependent on the model. The backend is implemented in Java and contains static as well as dynamic code. However, it is completely generated. The backend provides a server which offers functionality such as data storage and communication across apps.

Each of these three parts of the code is described in detail in the following.

% start of section for static map.apps implementation

\subsection{Static \mapapps Implementation}

The static \mapapps code is split into several bundles, which are used by the generated \mapapps apps. These bundles are located at \lstinline!src/main/js/bundles! and are explained in the following subsections.

\subsubsection{Form Controls}

The form controls are defined within the bundle \lstinline!md2_formcontrols!. The bundle uses and extends the existing \mapapps bundle \lstinline|dataform| with additional form elements. Each factory defined within the bundle of \MD specifies how a JavaScript-object can be transformed to a data form widget. To define your own dataform or to understand the concepts of a dataform component, the \href{http://developernetwork.conterra.de/documentation/31/developers/dataform}{\mapapps documentation} is a good place to start.

\begin{description}
	\item[DateTimeBoxFactory] Defines a form control for the component \lstinline|DateTimeInput|, which is identified by the keyword \lstinline|datetimebox|. The widget shows a view element that displays the time and the date of a \lstinline|datetime| value.
	\item[GridPanelFactory] Defines a form control for the component \lstinline|GridLayoutPane|, which is identified by the keywords \lstinline|md2gridpanel| and \lstinline|gridpanel|. The widget enables to structure multiple view elements in a grid.
	\item[ImageFactory] Defines a form control for the component \lstinline|Image|, which is identified by the keyword \lstinline|image|. The widget is able to display a static image within your app.
	\item[SpacerFactory] Defines a form control for the component \lstinline|Spacer|, which is identified by the keyword \lstinline|spacer|. A spacer sets whitespace between components or within the grid of a \lstinline|GridLayoutPane|.
	\item[StackContainerFactory] Defines a form control for the component \lstinline|AlternativesPane|, which is identified by the keyword \lstinline|stackcontainer|. This widget encapsulates the stack container within \href{http://dojotoolkit.org/reference-guide/1.10/dijit/layout/StackContainer.html}{\lstinline|dijit/ layout/StackContainer|}. It provides a view element which has multiple views, but shows only one, similar to a book or a slide show. The user can navigate between them using specific keys. 
	\item[TextOutputFactory] Defines a form control for the component \lstinline|Label|, which is identified by the keyword \lstinline|textoutput|. This widget enables to display non-editable text.
	\item[TooltipFactory] Defines a form control for the component \lstinline|Tooltip|, which is identified by the keyword \lstinline|tooltipicon|. This widget offers a tooltip behind a question mark icon.
	\item[UploadImageOutputFactory] Defines a form control for the component \lstinline|UploadedImageOutput|, identified by the keyword \lstinline|uploadimageoutput|. The widget is able to display an image within your app, which is uploaded/specified by the user. 
\end{description}

Special dataform elements enable the use of uploaded files. The \lstinline|UploadedImageOutput| displays images which have been uploaded by an app's user using a \lstinline|FileUpload| input element. Given that both elements are mapped to an entity's attribute of type \lstinline|file|, these elements retrieve an image from, or store an image on the server, respectively. For this procedure, a specialised \lstinline|remoteConnection| needs to be defined by the modeller, thus defining the remote location of this service and a local path where this service is able to store files (\lstinline|fileUploadConnection|). Consequently, this remote connection can be different from every other remote connection used, \eg by content providers.

\begin{figure}[b!]
\centering % trim=l b r t
\includegraphics[clip,  trim=0 17.4cm 3.cm 0, scale=0.8]{Fig/upload-sequence.pdf}
\caption{Procedure of uploading and retrieving user-uploaded images}
\label{fig:remoteFileUpload}
\end{figure}

As a further consequence, the uploaded file is not directly stored in the database. Instead, when called by a \lstinline|FileUpload| element, the upload service stores the file on disk at the specified path (\lstinline|storagePath|) and returns an identifier string of this file to the calling client (cf. \Cref{fig:remoteFileUpload}, step 1).
This identifier is then used throughout the model, particularly as the value of a corresponding attribute of type \lstinline|file|  (cf. \Cref{fig:remoteFileUpload}, step 2).

When such an identifier is encountered as the value of an \lstinline|UploadedImageOutput|, this element calls a servlet at the \lstinline|fileUploadConnection|, passing the identifier as a parameter (cf. \Cref{fig:remoteFileUpload}, step 4). That way, the image is downloaded for the client just as soon as it is needed, instead of loading it at every initialisation of entities by a content provider. Note that currently only JPEG images should be uploaded, since the servlet always tries to output any given file using the content type \lstinline|image/jpeg|.



\subsubsection{List of Open Issues}
\label{sec:listOfOpenIssues}

The \lstinline!md2_list_of_open_issues! comprises all code necessary to display the list of open issues within the app. This list shows all workflow instances, whose state is at a workflow element that belongs to the current app. Currently, the data listed in this widget are the the workflow element name, the last fired event and a timestamp for the last modification of the workflow state. 
The list is included as \lstinline!dijit_Widget! and is listed as a \lstinline!Tool! in the app.json under the bundle specifications of the \lstinline!toolset!. In the \lstinline!ListOfOpenIssues!\lstinline!Controller! a \lstinline!DataView! is created, which uses the workflow store as a \lstinline!DataViewModel!. The workflow store is described in \Cref{workflow_store}.
Workflow instances are not only listed, but it is possible to start the workflow element by clicking on the respective entry. Then, the \lstinline!ListOfOpenIssuesController! handles the event \lstinline!onClicked! and calls the function \lstinline!startWorkflow! of the respective \lstinline!MD2MainWidget!. The workflow instance ID is retrieved in combination with the content provider IDs of its current state. With these, the content provider referenced in the \lstinline!workflowStateHandler! are set to their respective values.

\subsubsection{Local Store} \label{local_store}



The local store within the bundle \lstinline|md2_local_store| is one of three stores used in the context of \mapapps within \MD. This store implements some of the guidelines from the \lstinline|dojo/Store| interface, which means, that it offers the methods \lstinline|query|, \lstinline|get|, \lstinline|put|, \lstinline|add|, and \lstinline|remove|.
The local store can be used by a content provider (set to local within the controller model). This store saves all data as cookies in the browser. Thus, the store is not meant for consistent data storage.

\subsubsection{Location Service}
The \lstinline|LocationStoreFactory| provides methods to convert a longitude and latitude pair into an address (i.e. a country, city, street, postal) and vice versa. In the first case, the method \lstinline|_getAddressForLocation| is used, which takes two parameters for the longitude and latitude value. In the second case, the method \lstinline|_getLocationForAddress| is used. This method takes a single string as input, which contains all the address information (\eg \textit{Schlossplatz M\"unster 48149}). For both methods, the result is a JavaScript object.

ArcGIS is the underlying API that is used for this (reverse-) geocoding. The URL to use this service is specified in the \lstinline|manifest.json| of that bundle. Currently, the URL is:

 \lstinline|http://geocode.arcgis.com/arcgis/rest/services/World/GeocodeServer|. 

\subsubsection{Runtime}

This bundle contains the main logic of the \MD \mapapps framework, which is mainly based on the \lstinline|MD2MainWidget| object. Most other sub bundles just enhance the functions of this widget.

\paragraph{MD2MainWidget}
The \lstinline!MD2MainWidget! is for example responsible for the opening and closing of views. 
Each workflow element (see \Cref{sec:developAndDeployMultiApps}) has its own instance of a \MD main widget. This is specified in the respective controller of the workflow element bundle inside the app. That is, the \lstinline!manifest.json! of the workflow element bundle references an \lstinline!_md2AppWidget! for its controller. Once the controller is activated (\ie the \lstinline!activate! function is called), the respective \MD main widget instance is built. This \MD main widget is implemented in the file \lstinline!MD2MainWidget!, which serves as the basic starting point to start a workflow. Thus, it provides methods to start a workflow element. There are different ways a workflow can be started:
\begin{itemize}
	\item A workflow can be started directly from the \mapapps tool bar. Each workflow element marked as \lstinline|startable| in the model gets its own tool. After clicking on it, the method \lstinline|startWorkflowFrom|\lstinline|Tool| is started, which first resets all current workflow elements and the workflow handlers and then calls the \lstinline|startWorkflow| function.
	\item A workflow can also be started from the list of open issues (see section \ref{sec:listOfOpenIssues}). Therefore, the \lstinline|startWorkflow| is directly started from the \lstinline|OpenIssuesListController|.
\end{itemize}
When the workflow element window is closed and the same operation to start a workflow element is called again (reopening the same entry of the list of open issues or restart the same tool) the same window with all data entered before will be opened.
However, all changes are dropped if another workflow instance is opened in between!
Each \lstinline|MD2MainWidget| contains a runtime variable \$, which contains important objects needed within the context of the widget. While many objects are created anew for each widget, some objects such as the \lstinline|WorkflowStateHandler| (cf. \Cref{par:workflowStateHandler}) are globally used and thus should only be created once. Therefore, many of these objects are injected as singletons within the \lstinline|mainfest.json| of the bundle \lstinline|md2_runtime|.
\Cref{fig:InitMD2MainWidget} depicts this initialization process.

\begin{figure}
\centering
\begin{tikzpicture}[
	redbox/.style = {rectangle, fill=ercisred, text =white, draw=none, text centered, drop shadow},
	arrowlbl/.style = {pos=.5, black},
	>=stealth
]

\draw (-3.5,-2.25) rectangle (1.2,-0.5);
\draw [fill=gray!20!white] (-3.5,-0.95) rectangle (-0.7,-0.5) node[pos=.5] {\small{List of open issues}};
\draw [redbox] (-3.3,-2) rectangle (1,-1) node[pos=.5] {OpenIssueListController};

\draw [->, thick, ercisred] (1,-1.1) -- (4.6,-1.1) node[above, arrowlbl] {\lstinline!startWorkflow!} -- (4.6,0.4) -- (6.5,0.4);

\draw [->, thick, ercisred] (1,-1.9) -- node[above, arrowlbl] {\lstinline!getMD2MainWidget!} (6,-1.9);

\node[draw] (tool) at (-1.25, -3) {\small{\textit{map.apps tool}}};
\draw [->, thick, ercisred] (tool) -- (5.35,-3) node[above, arrowlbl] {\lstinline!startWorkflowFromTool!} -- (5.35,0.2) -- (6.5,0.2);

\draw [->, thick, ercisred] (-2.9,1) -- node[above, arrowlbl] {\lstinline!load!} (-1.9,1);

\node [above] at (-0.9,0.5) {\lstinline!activate!};
\draw [->, thick, ercisred] (-1.8,1) -- (-1.8,0.5) -- (0,0.5);
\draw [fill=gray!20!white](-1.9,1.1) rectangle (-0.8,1.5) node[pos=.5] {\small{WfE}};
\draw (-1.9,-0.25) rectangle (2.25,1.5);
\draw [redbox] (0,0) rectangle (2,1) node[pos=.5] {Controller};

\draw [->, thick, ercisred] (2,0.65) -- node[above, arrowlbl] {\lstinline!build!} (6.5,0.65);

\draw [fill=gray!20!white] (5.7,1.1) rectangle (8,1.5) node[pos=.5] {\small{MD2Runtime}};
\draw (5.7,-2.25) rectangle (11.25,1.5);
\draw [redbox] (6.5,0) rectangle (11,1) node[pos=.5] {MD2MainWidget};
%\node [rotate=270] at (10.7,0.35) {\small{startWorkflow}};
%\draw [->, thick, ercisred] (10,0.8) -- (10.5,0.8) -- (10.5,0.2) -- (10,0.2);

\draw [->, thick, ercisred] (7,0) -- node[right, arrowlbl] {\lstinline!register!} (7,-1);

\draw [redbox] (6,-2) rectangle (11,-1) node[pos=.5] {WorkflowStateHandler};

\end{tikzpicture}
\caption{Initialization of the \MD main widget in order to start a workflow}
\label{fig:InitMD2MainWidget}
\end{figure}

\paragraph{Actions}
Each action type defined in the \MD DSL must also exist in the \MD runtime bundle. Individual actions are stored in the subfolder \lstinline!simpleactions!. Moreover, an \lstinline!ActionFactory! must provide a method that returns an instance of the respective action (\eg \lstinline!getLocationAction! for the \lstinline!LocationAction!). All actions provide a method \lstinline!execute! that implements the action. An individual constructor allows to initialize the action, \eg setting a city's name for a \lstinline!LocationAction!. The \lstinline!Action!\lstinline!Factory! is instantiated in the method \lstinline!build! of the \lstinline!MD2MainWidget! (cf. \Cref{fig:InitMD2MainWidget}). Thus, every workflow element can access and use this factory.

\paragraph{Content Provider}\label{par:contentProvider}
The content providers are responsible for saving and persisting the state of one or multiple objects. While the generated code only provides factories for the creation of content providers, the actual code is located in the static bundles. Each content provider is instantiated with a unique name, the app ID it belongs to, a store (either remote or local), the information whether it is a provider for a list of objects, a filter restricting the queried items and the information whether it is a remote or a local store.
Besides functions to get or set the content of the provider, functions are offered to access the injected store and persist the data.
Additionally, each content provider can inform other components of the app about changes within attributes. This can be used for example to refresh the values of view elements.
The function \lstinline|restore| and \lstinline|reset| are used within the \lstinline|WorkflowStateHandler| to  influence the state of the content provider, for example when the workflow instance is changed.

\paragraph{Data Mapper}
Whenever a value within a content provider changes it is necessary to inform the view elements to be able to refresh them. This is one within the classes of this folder.

\paragraph{Data Types}
In this folder all data types known by the \MD \mapapps framework are listed. Each data type provides additional functions for working with the data objects such as cast or compare operations. The \lstinline|TypeFactory| is used to instantiate an object according to its data type and is injected to the runtime variable \$ within the \lstinline|MD2MainWidget|.

\paragraph{Entities}\label{par:Entities}
This folder contains all internal entities known to the \MD \mapapps framework. Currently, this is the \lstinline|Location| entity and the abstract class \lstinline|_Entity|, which is inherited by all generated entities.

\paragraph{Events}
In the \MD modelling language it is possible to define actions based on events. Examples for such events are changes or clicks on view elements. To be able to map this behaviour in \mapapps, each possible events has its own class which subscribes a topic associated with the type of event it represents. The \lstinline|EventRegistry| has a list of all possible events and the root event classes enable to register actions to the specific events.

\paragraph{Handler}
This folder contains handlers for global events. These mainly display results in info or warning messages. Certain actions are bound to events by subscribing to topics. One example is the data action bound to the topic \lstinline[language=Javascript]|"md2/contentProvider/dataAction/${appId}"|. If any component is publishing a status (\lstinline[language=Javascript]|"success"| or \lstinline[language=Javascript]|"error"|) for an action (\eg \lstinline[language=Javascript]|"load"| or \lstinline[language=Javascript]|"save"|) a respective info message is shown in the lower right corner of the application.

\paragraph{Resources}
This folder contains images and style files.

\paragraph{Templates}
This folder contains the root html file of the \lstinline|MD2MainWidget|.

\paragraph{Validators}
In the model validators can be defined for view elements. This folder contains all existing validators, which can be created using the \lstinline|ValidatorFactory| which is injected in the runtime variable \$.

\paragraph{View}
The \lstinline|MD2MainWidget| is responsible for creating the views for the workflow elements. For this purpose it uses a \lstinline|ViewManager| which creates the view elements based on the \lstinline|view| entries within the manifest.json of the respective workflow element. The type of a view element is therefore linked to a dataform component either contained in the bundle \lstinline|md2_formcontrols| or within the external conterra bundle \lstinline|base/dataform|.

\paragraph{Workflow}\label{par:workflowStateHandler}
The \lstinline|WorkflowStateHandler| and the generated \lstinline|WorkflowEventHandler| are responsible together for managing the state of the current workflow instance. This includes the transitions between workflow elements within the context of one app as well as firing workflow events to the backend.  
For each started workflow instance a unique ID is generated and assigned to that instance. This is done in the method \lstinline!startWorfklow! of the \lstinline!MD2MainWidget!. A \lstinline!WorkflowStateHandler! provides methods to set and get the currently active workflow instance ID in a global context. This information is needed to suspend a workflow and to resume that workflow at a later time. The variable \lstinline|_lastStartedTool| provides information about how the workflow instance has been started. This is important when it comes to decide for a new \lstinline!startWorfklow! evaluation, if the current workflow instance has to be resumed or a new one should be created.

To change the current workflow element, the \lstinline|WorkflowStateHandler| provides the methods \lstinline|change|\lstinline|WorkflowElement| and \lstinline|fireEventToBackend|. 
The first one opens another \lstinline|MD2MainWidget|, as the workflow is continued in the same app.
The latter one will exit the workflow instance for the current app and close all windows, as the workflow is supposed to continue in another app. Additionally, the backend is informed about this step by calling the \lstinline|fireEventToBackend| method of the \lstinline|WorkflowStore| (cf. \Cref{workflow_store}). Since the \lstinline|WorkflowStore| is supposed to send the content provider IDs, this leads to a problem. 
The saving operation of the content providers is usually done shortly before the workflow event is fired. Since the content provider IDs are not set until the backend has answered to the web service calls, they may not be accessible yet. For this purpose, the \lstinline|WorkflowStateTransaction| is used. 
With each new workflow instance ID a new transaction is created. It keeps a list of all started content provider saving operations and only allows to fire a workflow event when no save operations are in progress. Additionally, the transaction is informed via a subscription about the termination of each content provider operation and will then retry to fire a workflow event, if one was queued before.

\subsubsection{Store}\label{store}

The \lstinline|md2_store| bundle provides the second of the three stores. It could also be called remote store, since it provides access to external data storage. It is again an implementation of the \lstinline|dojo/Store| and implements all necessary functions. The store is used within a content provider to query the current state of the objects belonging to the current workflow instance. In contrast to the local store, data which is saved within this store is persisted throughout the whole application landscape. For this purpose each store needs to be provided with an URL, pointing to the respective backend server. The data is then queried and stored using REST web services. One implementation of such a backend is automatically generated by the backend generator described in \Cref{sec:backend}. For the store to be able to find its respective backend, the URLs need to match each other (see \Cref{subsubsec:Backend}). 

\subsubsection{Workflow Store} \label{workflow_store}

The workflow store within \lstinline|md2_workflow_store| differs from the other stores provided by the static bundles. While the local (cf. \Cref{local_store}) the remote store (cf. \Cref{store}) are used within a content provider, the workflow store has its own purpose. It is used to save and query the status of all workflow instances. It is injected in the \lstinline|WorkflowStateTransaction| of the \lstinline|md2_runtime| bundle (cf. \Cref{par:workflowStateHandler}). There, it is responsible of informing the backend of any changes within the workflow instance. In the model this action can be represented by the \lstinline|FireEventAction|. Beside the information which event has been fired and which was the current workflow element and instance ID, the backend gets a list of all IDs registered in the content providers. This is necessary in order to be able to restore the state of the content providers in another app.

The workflow store is also referenced in the list of open issues. The workflow store implements, similar to the other stores, the functions of the \lstinline|dojo/Store|. This makes it possible to hand the workflow store over to a \lstinline|DataView|, which then displays the information retrieved from the store. The \lstinline|query| function enables to query the whole list of the list of open issues.

For the workflow store to work correctly, the app.json needs to provide an appropriate configuration for the REST location and the current app ID, for which the web service should filter. The following snippet (with equivalent values) is automatically generated:
\begin{lstlisting}[language=Javascript]
"md2_workflow_store": {
	"MD2WorkflowStore": {
		"uri": "http://localhost:8080/ReferenceProject.backend/service/workflowState/",
		"app": "Citizenapp"
	}
}
\end{lstlisting}

% end of section for static map.apps implementation

\subsection{Generated \mapapps Implementation} 

The generated code of the \mapapps applications contains everything, that can not be generalised in the static implementation. This includes mainly the defined content providers, entities, and workflow elements, as well as the workflow mechanism described in the workflow model. Additionally, some components of the static \mapapps implementation are injected with model specific settings.

It is important to note that each app definition within the workflow model will result in its own app. Each app gets its own content provider, model, and workflow element implementation. However, files for those objects are only created when they are necessary within the context of the specific app.

\subsubsection{Content Providers}
The content providers are created as an individual bundle within each app. Each content provider is represented within its own file. However, the file does not incorporate the content provider itself, but merely a content provider factory. It is used to create an instance of the content provider class included in the static \lstinline|md2_runtime| bundle (cf. \Cref{par:contentProvider}).

Each content provider is created using the name defined in the model, the app ID and an instance of the respective local or remote store. The stores are created using a store factory injected within the manifest.json.
Additionally, the remote stores are initialised with the URL of their corresponding remote connection.

Besides the content providers defined within the model, two local content providers are created (namely \lstinline|__returnStepStackProvider| and \lstinline|__processChainControllerStateProvider|). They enable the usage of process chains within a workflow element. Therefore, they save the steps taken and their respective state so that it is possible to return to a previous step.

\subsubsection{Models}

The entities are grouped within the bundle \lstinline|md2_models|. Each defined entity and enum in the model gets its own file. Additionally, the two entities \lstinline|__ProcessChainControllerState| and \lstinline|__returnStepStack| are created to map the state of the two content providers listed above. Each entity inherits from the static class \lstinline|_Entity| described in \Cref{par:Entities}. They describe their own data type as well as the types of all their attributes, or, in case of an enum, the possible values. Additionally, the entities have an \lstinline|_initialize| function to create empty attribute types for all attributes, except for the referenced entities. For those the values have to be set manually using a return value of another content provider later in the code.

\subsubsection{Workflow Elements}

For each workflow element of the app an individual bundle is created. This bundle specifies a \lstinline|Controller|, which is the instance factory for a \lstinline|MD2MainWidget|. The \lstinline|MD2MainWidget| itself is further specified within the manifest.json. Here, all view elements are defined in addition to the app ID, the workflow element ID, the webserviceBackendURI, the window title, and the action called upon initialization of the workflow element. This \lstinline|onInitialized| action is executed by the \lstinline|MD2MainWidget| when opening the widget. Besides the \lstinline|Controller|, all actions which inherit from the static class \lstinline|_Action| are specified. The \lstinline|CustomActions.js| implements a class that serves as an instance factory for all actions defined within the workflow element.

\subsubsection{Workflow}
This bundle contains the \lstinline|WorkflowEventHandler|, which keeps a list of all  \lstinline|MD2MainWidget|. Additionally, it contains the specification for which workflow follows which, and for the differentiation between the case when an event has to be fired to the backend or a workflow element must be changed within the app. After exiting the context of a workflow instance, the event handler has the capability to reset all \lstinline|MD2MainWidget| instances registered in order to be prepared for a new instance.

\subsubsection{App.json} \label{subsubsec:mapapps-generated-appjson}
The app.json is used to inject further information into the static bundles. These settings are mainly backend URLs and app IDs, which are needed in this components.
Besides that, the app.json  contains a list of all bundles used within the app context.
