% !TeX spellcheck = en_GB
% !TeX program = xelatex
% !TeX root = ../md2-user-handbook.tex

The current implementation of the \MD framework generates web-based apps for a framework called map.apps, which is mainly based on JavaScript. Code generation of Android and iOS apps are also targeted, but not fully implemented yet.

The generated code for map.apps can be subdivided into three parts: static map.apps code, dynamically generated map.apps code and a backend. The static map.apps code contains the part of the code which does not depend on the models created in the \MD DSL. Since it is static, it does not need to be generated, but is required for the overall functionality of the generated apps. The dynamically generated part is completely dependent on the model. The backend is implemented in Java and contains static as well as dynamic code. However, it is completely generated. The backend provides a server which offers functionality such as data storage and communication accross apps.

Each of these three parts of the code is described in detail in the following.

% start of section for static map.apps implementation

\subsection{Static map.apps Implementation}

The static map.apps code is split into several bundles, which are then used by the generated map.apps code. These bundles are located at \texttt{src/main/js/bundles} and are explained in the following subsections.

\subsubsection{Form controls}

The form controls are defined within the bundle md2\_formcontrols. It extends the existing map.apps framework with additional form elements, which can be used in \MD. 

TODO: Description of different form controls.

\subsubsection{List of open issues}

The \texttt{md2\_list\_of\_open\_issues} comprises all code necessary for the displaying the list of open issues within the app. This list shows all workflow instances, whose state is at a workflow element, which belongs to the current app. Currently supported data listed in this widget are the guid of the workflow instance, the workflow element name and the last fired event. 
The list is included as \texttt{dijit\_Widget} and is listed as a \texttt{Tool} within the app.json under the bundle specifications of \texttt{toolset}. In the \texttt{ListOfOpenIssuesController} a \texttt{DataView} is created, which uses the workflow store as a \texttt{DataViewModel}. The workflow store is described in \cref{workflow_store}.
In addition to just displaying the workflow instances it is possible to start the workflow element through clicking on the respective entry. Therefore the \texttt{ListOfOpenIssuesController} handles the event \texttt{onClicked} and calls the function \texttt{startWorkflow} of the respective \texttt{MD2MainWidget}.

\subsubsection{Local store}

\subsubsection{Location service}

\subsubsection{Runtime}
% actions
% contentprovider
% datamapper
% datatypes
% entities
% events
% handler
% legacy
% resources
% templates
% util
% validators
% view
% workflow

\textit{MD2MainWidget}

Each workflow element (see \ref{sec:developAndDeployMultiApps}) has its own instance of a \MD main widget. This is specified in the respective controller of the workflow element bundle inside the app. That is, the \texttt{manifest.json} of the workflow element bundle references an \texttt{\_md2AppWidget} for its controller. Once the controller is activated (i.e. the \texttt{activate} function is called), the respective \MD main widget instance is built. 

This \MD main widget is implemented in the file \texttt{MD2MainWidget}, which serves as the basic starting point to start a workflow. Thus, it provides methods to start a workflow element, e.g., from the map.apps tool bar.

\begin{figure}[htb!]
\centering
\begin{tikzpicture}[redbox/.style = {rectangle, fill=ercisred, text =white, text width=5em, minimum height = 3em, text centered, drop shadow}, >=stealth]
\node[redbox] (ctrl) {WfE Controller};
\node[redbox, right = of ctrl] (main) {\MD Main Widget};
\draw[->, thick, ercisred] (ctrl) -- node[above]{build} (main);

\end{tikzpicture}
\caption{Initializing of the \MD main widget}
\label{fig:MD2MainWidget}
\end{figure}

\subsubsection{Store}

\subsubsection{Workflow store} \label{workflow_store}

% end of section for static map.apps implementation

\subsection{Dynamic map.apps code}

\subsection{Backend}