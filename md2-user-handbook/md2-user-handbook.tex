% !TEX TS-program = XeLaTeX
% !TeX spellcheck = en_US
% !TeX program = xelatex


\documentclass[a4paper, 12pt, titlepage, headsepline, listof = totoc, bibliography = totoc, numbers = noenddot]{scrbook} %
\usepackage[left = 2cm, right = 2cm, top = 3cm, bottom = 3cm]{geometry} % spaces on the sides of the paper
%\usepackage[utf8]{inputenc} %pdflatex
\usepackage{polyglossia} %xelatex
\setdefaultlanguage[variant=british]{english} %xelatex

% fonts and layout (xelatex)
%\defaultfontfeatures{Mapping=tex-text}
\setmainfont[Mapping=tex-text]{Adobe Garamond}
\setsansfont[Mapping=tex-text]{Meta}
\setmonofont{Source Code Pro}
\linespread{1.3}
\frenchspacing
\usepackage{setspace}
\usepackage{microtype}
\usepackage{xspace}
\usepackage{enumitem}
\raggedbottom
\setlist[itemize]{noitemsep, topsep=0pt}

% Graphics, subfigures, and colours
\usepackage{graphicx}
\usepackage[bf,IT]{subfigure}% http://ctan.org/pkg/subfigure
\usepackage{tikz}
\usetikzlibrary{calc,positioning,shadows,arrows,shapes}
%\usetikzlibrary{calc,fit,shapes,shapes.callouts,shapes.symbols,positioning,shadows,arrows,decorations,backgrounds,plotmarks,external}
%\tikzexternalize[prefix=tikz/]

% Listings
\usepackage{listings}

\lstdefinelanguage{MD2}{
  morekeywords={WorkflowElement,WorkflowElements, fires, start, App, startable, appName, package, entity, string,
  date, float, enum, integer, boolean, time, datetime, GridLayoutPane, FlowLayoutPane, TabbedPane, TextInput,
  label, tooltip, type, default, timestamp, OptionInput, options, CheckBox, checked, Label, text, style, Tooltip,
  Button, Image, height, width, src, Spacer, AutoGenerator, exclude, only, textProposition, contentProvider,
  ContentProvider, EntitySelector, color, fontSize, textStyle, main, appVersion, modelVersion, workflowManager,
  defaultConnection, CustomAction, CombinedAction, CustomCodeFragement, SimpleAction, WebServiceCallAction, WebServiceCall, externalWebService, url, method, GET, POST, PUT, DELETE, queryparams, bodyparams, on, from, Container, Content,
  GlobalEventType, OnConditionEvent, elementEventType, GotoViewAction, Workflow, WorkflowStep, ContentProvider, use,
  for, to, latitude, longitude, altitude, citystreet, number, postalCode, country, province, ElementEventType, event,
  valid, empty, filled, and, equals, not, is, or, Condition, Boolean, bind, unbind, Validator, ContainerElement,
  ContentElement, validator, IsIntValidator, NotNullValidator, IsNumberValidator, IsDateValidator, RegExValidator,
  NumberRangeValidator, StringRangeValidator, map, unmap, call, actions, message, min, max, regEx, format, minLength,
  maxLength, RemoteValidator, connection, model, attributes, workflow, ProcessChainStep,step, view, forwardCondition,
  forwardMessage, backwardMessage, backwardCondition, forwardOnEvent, backwardOnEvent, forwardEvents, subProcessChain,
  remoteConnection, uri, ProcessChain, end, defaultProcessChain, onInit, FireEvent, ProcessChainProceed, ProcessChainReverse, ProcessChainGoto, SetProcessChain, GotoView, where, Location, Disable, Enable, DisplayMessage, ContentProviderOperation, ContentProviderReset, inputs, outputs, cityInput, streetInput, streetNumberInput, postalInput, countryInput,FileUpload, UploadedImageOutput, imgHeight,imgWidth,fileUploadConnection, storagePath, file, optional,
  latitudeOutput, longitudeOutput, name, description, proceed, reverse, goto, return, given, do, returnTo, disabled,
  action, if, else, set, elseif,
  invokable, at, using, as},
  otherkeywords={},
  sensitive=true,
  morecomment=[l]{//},
  morecomment=[n]{/*}{*/},
  morestring=[b]",
  morestring=[b]',
  morestring=[b]"""
}


\lstdefinelanguage{JavaScript}{
	keywords={typeof, new, true, false, catch, function, return, null, catch, switch, var, if, in, while, do, else, case, break},
	keywordstyle=\color{blue}\bfseries,
	ndkeywords={class, export, boolean, throw, implements, import, this},
	ndkeywordstyle=\color{darkgray}\bfseries,
	identifierstyle=\color{black},
	sensitive=false,
	comment=[l]{//},
	morecomment=[s]{/*}{*/},
	commentstyle=\color{purple}\ttfamily,
	stringstyle=\color{brown}\ttfamily,
	morestring=[b]',
	morestring=[b]"
}

\definecolor{javastringblue}{rgb}{0.164,0,1} % for strings
\definecolor{javagreen}{rgb}{0.25,0.5,0.35} % comments
\definecolor{javapurple}{rgb}{0.5,0,0.35} % keywords
\definecolor{javadocblue}{rgb}{0.25,0.35,0.75} % javadoc

\lstdefinelanguage{Java}{
	keywordstyle=\color{javapurple}\bfseries,
	stringstyle=\color{javastringblue},
	commentstyle=\color{javagreen},
	morecomment=[s][\color{javadocblue}]{/**}{*/},
}
\lstdefinelanguage{Simple}{
}

%Set default language
\lstset{
	language=MD2,
	aboveskip=3mm,
	belowskip=3mm,
	showstringspaces=false,
	columns=flexible,
	basicstyle={\footnotesize\ttfamily},
	numbers=none,
	numberstyle=\tiny\color{gray},
	keywordstyle=\color{mauve},
	commentstyle=\color{dkgreen},
	stringstyle=\color{blue},
	breaklines=true,
	breakatwhitespace=true
	tabsize=3
}
%\usepackage{caption} 
\renewcommand{\thesubfigure}{\alph{subfigure}}% (a) -> a
%\usepackage[table]{xcolor}
\definecolor{Lightgray}{gray}{0.9}
\definecolor{dkgreen}{rgb}{0,0.6,0}
\definecolor{gray}{rgb}{0.5,0.5,0.5}
\definecolor{mauve}{RGB}{127,0,85}

% Define ERCIS colors (see official corporate design manual)
\definecolor{ercisblack}{RGB}{0,0,0}
\definecolor{ercisgrey}{RGB}{94,94,93}
\definecolor{ercisred}{RGB}{133,35,57}
\definecolor{ercislightred}{RGB}{200,156,166}
\definecolor{ercisblue}{RGB}{135,151,163}
\definecolor{ercisclaim}{named}{ercislightred}

% Additional ERCIS colors
\definecolor{ercisdarkblue}{RGB}{67,92,139}
\definecolor{erciscyan}{RGB}{0,156,179}
\definecolor{ercisorange}{RGB}{231,124,18}
\definecolor{ercisgreen}{RGB}{135,191,42}

% Define WWU colors (see official Corporate Design manual, p. 21)
\definecolor{pantoneblack7}{RGB}{62,62,59}

%maths
\usepackage{amssymb}
\usepackage{amsfonts}
\usepackage{amsmath}


%STOP ROMAN NUMBERS FROM GROWING TO THE RIGHT IN TAB OF FIGURES!!!!!!11
\makeatletter \def\@pnumwidth{3em}\makeatother 


% notes, reminder
\setlength{\marginparwidth}{1.5cm} % width of todo note
\usepackage[ngerman,textwidth=1\marginparwidth,textsize=scriptsize,backgroundcolor=yellow,linecolor=blue]{todonotes} %,disable
%\reversemarginpar % put todo note left, where the margin is (way) larger (disabled, since it's not)

%Acronyms
%\usepackage[printonlyused]{acronym}

%Quotations
\usepackage{csquotes}
\setquotestyle[american]{english}

% References Section
%\usepackage[backend=biber,maxbibnames=99,style=alphabetic,firstinits=true]{biblatex} %,uniquename=allfull
%\bibliography{lit.bib} % Which file to use for the bibliography?


% hyperlinks and references
\usepackage{hyperref}
\usepackage{cleveref}



% Table design
\usepackage{supertabular}
\usepackage{booktabs}
\usepackage{float}
\restylefloat{table}

% Clubs and Widows (Hurenkinder und Schusterjungen)
\clubpenalty = 10000 
\widowpenalty = 10000 
\displaywidowpenalty = 10000

%%%%%%%%%%%%%%%%%%%%%%%%%%%%%%%%%%%%%%%%%%%%%%%%%%%%%%%%%%%%%%%%
%% design of the head of the report pages

\usepackage{scrpage2}
\clearscrheadings                 		% clears all predefined formats
\pagestyle{scrheadings}			% use this style only on the actual text
\ohead{}		% writes your name on each side in the upper right corner
\automark{section}                  
\ihead{\headmark}				% automatically writtes the section name in the upper left corner
\cfoot{\pagemark}				% page number on the bottom (center)


\newcommand{\MD}{MD$^2$\xspace}
\newcommand{\mapapps}{map.apps\xspace}
%%%%%%%%%%%%%%%%%%%%%%%%%%%%%%%%%%%%%%%%%%%%%%%%%%%%%%%%%%%%%%%%
%% cover sheet

\title{User's Handbook}
\subtitle{map.apps with MD2}
\author{Project Seminar\\
Model-driven Mobile Development\\
(University of Münster)}
%\date{January 18, 2015}

% hyperref PDF metadata
\hypersetup{pdfinfo={
	Title={Handbook map.apps with MD2},
	Author={PS MD2 (University of Münster)}
}}


\begin{document}



\thispagestyle{empty}
%\maketitle
\begin{titlepage}

\newcommand{\HRule}{\rule{\linewidth}{0.5mm}} % Defines a new command for the horizontal lines, change thickness here

\center % Center everything on the page
 
%-------------
%   LOGOS
%-------------

\begin{minipage}[b]{0.5\textwidth}
\begin{flushleft}
\includegraphics[width=0.9\textwidth]{Fig/wwu-logo-title}
\end{flushleft}
\end{minipage}
~
\begin{minipage}[b]{0.4\textwidth}
\begin{flushright}
\includegraphics[width=0.4\textwidth]{Fig/md2-logo}
\end{flushright}
\end{minipage}\\[1.5cm] 
 
 
 
%----------------------------------------------------------------------------------------
%	HEADING SECTIONS
%----------------------------------------------------------------------------------------

\textsc{\LARGE Project Seminar\\ Model-driven Mobile Development}\\[1.5cm] % Name of your university/college
%\textsc{\Large Major Heading}\\[0.5cm] % Major heading such as course name
%\textsc{\large Minor Heading}\\[0.5cm] % Minor heading such as course title

%----------------------------------------------------------------------------------------
%	TITLE SECTION
%----------------------------------------------------------------------------------------

\HRule \\[0.8cm]
{ \huge \bfseries MD$\mathbf{^2}$ -- Handbook}\\[0.4cm] % Title of your document
\HRule \\[3cm]
 
%----------------------------------------------------------------------------------------
%	AUTHOR SECTION
%----------------------------------------------------------------------------------------

\begin{minipage}[t]{0.4\textwidth}
\begin{flushleft} \large
\emph{Group Members:}\\
Jan Christoph Dageförde\\
Julia Dittmer\\
Andreas Fuchs\\
Carolin Gülpen\\
Holger Koelmann\\
Malte Möser\\
Tobias Reischmann
\end{flushleft}
\end{minipage}
~
\begin{minipage}[t]{0.5\textwidth}
\begin{flushright}\large

\emph{Supervisors:}\\
Prof. Dr. Herbert Kuchen\\
Jan Ernsting\\\bigskip

Group for Practical Computer Science \\
University of Münster\\
\end{flushright}
\end{minipage}\\[3cm]

% If you don't want a supervisor, uncomment the two lines below and remove the section above
%\Large \emph{Author:}\\
%John \textsc{Smith}\\[3cm] % Your name

%----------------------------------------------------------------------------------------
%	DATE SECTION
%----------------------------------------------------------------------------------------

{\large \today}\\[3cm] % Date, change the \today to a set date if you want to be precise

%----------------------------------------------------------------------------------------
%	LOGO SECTION
%----------------------------------------------------------------------------------------

%\includegraphics{Logo}\\[1cm] % Include a department/university logo - this will require the graphicx package
 
%----------------------------------------------------------------------------------------

\vfill % Fill the rest of the page with whitespace

\end{titlepage}

%%%%%%%%%%%%%%%%%%%%%%%%%%%%%%%%%%%%%%%%%%%%%%%%%%%%%%%%%%%%%%%%
%% table of contents

\thispagestyle{empty}
%\bgroup
%\begin{spacing}{0.87}  % Damit alles auf eine Seite passt
\tableofcontents
%\end{spacing}
%\egroup
%%%%%%%%%%%%%%%%%%%%%%%%%%%%%%%%%%%%%%%%%%%%%%%%%%%%%%%%%%%%%%%%
%% the document itself


\clearpage
\setcounter{page}{1}
% Provides commands for chapter structure having no numbering on highest layer
\newcommand*\theaddchap[1]{
	\addchap{#1}
	\setcounter{table}{0}
	\setcounter{figure}{0}
	\setcounter{section}{0}
}
\renewcommand*\thesection{\arabic{section}}
\renewcommand*\thetable{\arabic{table}}
\renewcommand*\thefigure{\arabic{figure}}
\setcounter{secnumdepth}{5}

\theaddchap{Introduction}
\todo{Searchtags for TODOs: @TOBI, @JAN, @CARO, @MALTE, @JULE, @ANDI, @HOLGER; \newline \newline FOR EVERYONE: @ALL}
% !TeX spellcheck = en_GB
% !TeX program = xelatex
% !TeX root = md2-user-handbook.tex

\todo{@ALL review pls}

The goal of this chapter is to provide the reader with the detailed information necessary for the further development of the \MD framework. It includes the required information about the 

\begin{itemize}
	\item installation procedure,
	\item DSL Semantics,
	\item structure of the \mapapps and backend implementation
	\item and  gives insights about the structure and interaction of the generated and static code.
\end{itemize}

In addition, it is also noteworthy that while further developing the framework, the development process should consider that changes within the DSL might also have a direct effect on validators and the formatter. This should be kept in mind during the planning and implementation of future changes.

\theaddchap{Modeler's Handbook}
\label{cha:modelersHandbook}

% !TeX spellcheck = en_GB
% !TeX program = xelatex
% !TeX root = md2-user-handbook.tex

\todo{@ALL review pls}

The goal of this chapter is to provide the reader with the detailed information necessary for the further development of the \MD framework. It includes the required information about the 

\begin{itemize}
	\item installation procedure,
	\item DSL Semantics,
	\item structure of the \mapapps and backend implementation
	\item and  gives insights about the structure and interaction of the generated and static code.
\end{itemize}

In addition, it is also noteworthy that while further developing the framework, the development process should consider that changes within the DSL might also have a direct effect on validators and the formatter. This should be kept in mind during the planning and implementation of future changes.

\section{Installation}
\label{sec: Installation}
% !TeX spellcheck = en_GB
% !TeX program = xelatex
% !TeX root = md2-user-handbook.tex



The following software is required prior to the next steps:

\begin{itemize}
\item A current Eclipse IDE with support for Java EE development (e.g. Luna)
\item Using the \href{https://eclipse.org/Xtext/download.html}{Xtext Update Site}, install a current version of the Xtext redistributable
\item From the archive that you obtained, install the MD2 features
\item GlassFish 4.+
\item map.apps 3.1.0
\item NetBeans EE (e.g. Version 8.0)
\item Apache Tomcat 7.0 with running map.apps runtime \todo{reference map.apps installation guide}
\end{itemize}



\todo{Den MapApps installation Guide hier einbinden oder im Anhang referenzieren? @ALL}




\section{Getting Started}
\label{sec: Getting Started}
% !TeX spellcheck = en_GB
% !TeX program = xelatex
% !TeX root = md2-mapapps-installation-guide.tex

\subsection{Developing a Single App} 
\label{subsec:SingleAppDev}

\subsubsection{Workflow} 
\label{subsubsec:Workflow}

\subsubsection{Model} 
\label{subsubsec:Model}

\subsubsection{View} 
\label{subsubsec:View}

\subsubsection{Controller} 
\label{subsubsec:Controller}


\subsection{Deploying a Single App}
\label{subsec:SingleAppDev}

\subsubsection{Backend} 
\label{subsubsec:Backend}

\subsubsection{map.apps} 
\label{subsubsec:mapapps}


\section{Development and Deployment of Multiple Apps}
\label{sec:developAndDeployMultiApps}
% !TeX spellcheck = en_GB
% !TeX program = xelatex
% !TEX root = ../md2-user-handbook.tex

The \MD framework allows to model and generate workflows that involve multiple apps. For this purpose, several apps rather than just one can be specified in the workflow layer. These apps can share the same workflow elements or use different ones as shown in Listing \ref{lst:multipleApps}. Apart from that, the workflow will look as usual, with the sequence of workflow elements being determined via events. Each app is provided with a list of open issues in the start screen. In this list, all events are presented that were fired from another app and are supposed to start a workflow element which belongs to the current app. A user can simply click on a listed issue to continue the workflow in the appropriate workflow element.

\todo{Why do we specify a label here, instead of just using the code block as in the previous sections?}
\begin{lstlisting}[language=MD2, label=lst:multipleApps, caption=Workflow definition for multiple apps]
package <ProjectName>.workflows

WorkflowElement <WorkflowElementOne>
<...>
WorkflowElement <WorkflowElementTwo>
<...>
WorkflowElement <WorkflowElementThree>
<...>

App <AppID1> {
	<WorkflowElementOne> (startable: STRING),
	<WorkflowElementTwo>
}

App <AppID2> {
	<WorkflowElementOne>,
	<WorkflowElementThree>
}
\end{lstlisting}

The deployment of multiple apps is similar to that of a single app. In this case, however, not just one but all created apps have to be deployed, \eg by setting the corresponding symbolic links as described in  \Cref{subsec:SingleAppDep} for each app.


\section{Additional Features}
\label{sec:Additional Features}
% !TeX spellcheck = en_GB
% !TeX program = xelatex
% !TeX root = md2-user-handbook.tex

\subsection{Uploading and Displaying Files}
\label{subsec: UploadSaveDisplayWebServices}
The current \MD version allows for uploading and displaying files such as images to or in an application. Therefore, in addition to conventional data types (e.g. string) the DSL comprises a data type representing files. In the model file of a \MD model this type can be assigned to an attribute, which can be marked as optional (cf. listing X). Moreover, in the view file the input element for files -- the FileUpload construct -- needs to be used. Using this construct a button having the specified attributes and allowing for uploading a file, is displayed on the respective UI form. In the controller file a remote connection specifying the location to which the file should be uploaded needs to be defined and listed in the main block of the controller. 

Aside from uploading files, they can be displayed in an application which is done by using the UploadedImageOutput construct in the view file.

Moreover, the content view elements should be linked to content provider fields in the init action to establish a connection between the UI elements and the respective data fields, as it is applicable for all other view content elements.

A specification of a sample image upload and its display, including all applicable attributes, is depicted in the following listing.

\begin{lstlisting}

//in model file:
entity sampleEntity {
	picture : file (optional)
}

// in view file:
// provide upload button
FileUpload pic {
	label "piclabel"
	text "pictext"
	tooltip "pictooltip"		
	style picstyle
	width 42%
}		

// display image (here: image uploaded before)
UploadedImageOutput pic1display {
	imgHeight 2
	imgWidth 2
	width 2
}

// in controller file:
main {
	//...
	fileUploadConnection FileUploadConnection
}

remoteConnection FileUploadConnection{
	uri "http://localhost:9090/proxy?sampleUri"
	storagePath "sampleFileUploadPath"
}

// moreover: specify respective content provider and mappings in init action

\end{lstlisting}




\subsection{Calling RESTful Web Service from the App}
\label{subsec: CallingWebServices}
With the current \MD version it is possible to call RESTful web services that are provided by external applications. To do so, it is necessary to specify the web service's url and REST method (e.g. GET), as well as the parameters to be transferred to it. The parameters are represented as <key, value> pairs and can be sent as query parameters and/or via the body of the request. Accordingly, depending on the option expected by the service to be called, the DSL allows the modeler to specify queryparams or bodyparams. 

Aside from static values to be set at design time, it is possible to set a parameter to the value of a particular ContentProviderPath, i.e. the value of a content provider's field, which is derived at run time. If the value is set statically at design time the data types String, Integer, Float and Boolean are allowed. 

An exemplary web service description based on the DSL as well as the corresponding call of the action is depicted in the following.

\begin{lstlisting}

// Specification of the web service call
externalWebService sampleWebService {
	url "http://sampleURL"
	method POST
	queryparams(
		"param1": 42	
	)
	bodyparams (
		"param2": "sampleString"
		"param3": sampleProvider.sampleField
	)
}

// Specify action to call the web service
action CustomAction callWS {
	call WebServiceCall sampleWebService
}
	
\end{lstlisting}





\subsection{Control of Workflow by calling a RESTful Web Service}
\label{subsec: WorkflowControlThroughWS}


\theaddchap{Developer's Handbook}
\todo{@ALL what about the preprocessor [it is after all the basis for all the generators?]}
% !TeX spellcheck = en_GB
% !TeX program = xelatex
% !TeX root = md2-user-handbook.tex

\todo{@ALL review pls}

The goal of this chapter is to provide the reader with the detailed information necessary for the further development of the \MD framework. It includes the required information about the 

\begin{itemize}
	\item installation procedure,
	\item DSL Semantics,
	\item structure of the \mapapps and backend implementation
	\item and  gives insights about the structure and interaction of the generated and static code.
\end{itemize}

In addition, it is also noteworthy that while further developing the framework, the development process should consider that changes within the DSL might also have a direct effect on validators and the formatter. This should be kept in mind during the planning and implementation of future changes.

\section{Installation}
% !TeX spellcheck = en_GB
% !TeX program = xelatex
% !TeX root = md2-user-handbook.tex



The following software is required prior to the next steps:

\begin{itemize}
\item A current Eclipse IDE with support for Java EE development (e.g. Luna)
\item Using the \href{https://eclipse.org/Xtext/download.html}{Xtext Update Site}, install a current version of the Xtext redistributable
\item From the archive that you obtained, install the MD2 features
\item GlassFish 4.+
\item map.apps 3.1.0
\item NetBeans EE (e.g. Version 8.0)
\item Apache Tomcat 7.0 with running map.apps runtime \todo{reference map.apps installation guide}
\end{itemize}



\todo{Den MapApps installation Guide hier einbinden oder im Anhang referenzieren? @ALL}




\section{DSL Semantics}
% !TEX root = ../md2-user-handbook.tex
\label{sec:dev-semantics}

The \MD framework is intended to provide a cross-platform solution, \ie to generate apps not only for \mapapps but also other platforms such as Android or iOS. For this purpose, this section delivers an overview about the semantics of the DSL, \eg the different patterns targeted or forms of communication that are implied by certain model constructs. This will enable future developers to generate apps for other platforms which provide the same functionality as the apps currently generated for \mapapps.

First of all, the MVC pattern with additional workflow layer used in the DSL should also be represented in the generated code.

\paragraph*{Workflow Layer}
The workflow layer defines different apps and their workflow elements. Since workflows bundle specific functionality, each app can be seen as a user role, and the assigned workflow elements represent the role's permissions. However, a sophisticated user or role management is not implemented in the \MD framework.

Every app is supposed to have a start screen which contains buttons for workflow elements that can be started in the app as well as a list of open issues (workflows in a specific state) that can be continued by the app. The belongingness of workflow elements to apps is represented in a map in the backend, which connects workflow elements to their apps. This is for example important for the determination of open issues which are allowed to be continued from a specific app.

Similar to that, the backend needs to know the sequence of workflow elements, \ie which workflow elements are to be started after which event and when to end the workflow. Note, that two workflow elements can fire the same event and start different workflow elements. Thus, the backend also needs to know which event/workflow element combination initializes the start of a specific new workflow element.

However, if a workflow element fires an event which starts a new workflow element within the same app, this should be handled by the app-specific event handler, so that no backend communication is required. This is important to allow temporary off-line usage of apps in the future. Thus, the backend handles the start of new workflows \textit{across} apps (currently implemented as EventHandlerWS) and the app-specific event handler is responsible to start new workflow elements \textit{within} apps.

When a workflow element is started across apps, it will appear in the list of open issues of all apps that have the respective workflow element assigned.

\paragraph*{Model Layer}
The model is a rather thin layer in the overall architecture, the only components contained are entities and enumerations. In order to access core data functionality, a data model has to be setup that defines the database to be accessed later on by the content providers. This database is currently located in the backend and should therefore be accessible by apps from all platforms. 

\paragraph*{Controller Layer}
The biggest and most important layer in this architecture is the controller layer, which has the role to connect the view with the model and vice versa. It consists of several workflow elements, each being an independent controller. The default process chain of a workflow element should be used as starting process chain. Likewise, the first view from this process chain should be used as start view for the workflow element. Each workflow element (\ie each controller) requires its own initialization, \eg mappings of content to views. The required actions for initializations can be found in the onInit block in a workflow element. When a workflow element fires a workflow event, it should be terminated and the control handed to the app-specific event handler or the backend as described for the workflow layer. 

Within the body of workflow elements, the controller behavior can be defined using actions and process chains. Process chains will be converted to actions in the preprocessor, and therefore do not require a generator for different platforms.

Content provider in the controller layer are used for data provision. Webservice-based communication to the backend is required for every platform in order to store and request the data.

\paragraph*{View Layer}

The view layer has not been changed during the course of this project seminar. View elements should be implemented with the functionality described in \Cref{subsubsec:View}.
 

\section{map.apps Implementation}
% !TEX root = ../md2-user-handbook.tex
% !TeX spellcheck = en_GB
% !TeX program = xelatex

\lstset{language=Simple}

\label{sec:dev-mapapps}

The current implementation of the \MD framework generates web-based apps for a framework called 
\mapapps, which is mainly written in JavaScript. Code generation for Android and iOS applications are also targeted, but not fully implemented yet.

The generated code for \mapapps can be subdivided into three parts: static \mapapps code, dynamically generated \mapapps code and a backend. The static \mapapps code contains the part of the code which does not depend on the models created in the \MD DSL. Since it is static, it does not need to be generated, but is required for the overall functionality of the generated apps. The dynamically generated part is dependent on the model. The backend is implemented in Java and contains static as well as dynamic code. However, it is completely generated. The backend provides a server which offers functionality such as data storage and communication across apps.

Each of these three parts of the code is described in detail in the following.

% start of section for static map.apps implementation

\subsection{Static \mapapps Implementation}

The static \mapapps code is split into several bundles, which are used by the generated \mapapps apps. These bundles are located at \lstinline!src/main/js/bundles! and are explained in the following subsections.

\subsubsection{Form Controls}

The form controls are defined within the bundle \lstinline!md2_formcontrols!. The bundle uses and extends the existing \mapapps bundle \lstinline|dataform| with additional form elements. Each factory defined within the bundle of \MD specifies how a JavaScript-object can be transformed to a data form widget. To define your own dataform or to understand the concepts of a dataform component, the \href{http://developernetwork.conterra.de/documentation/31/developers/dataform}{\mapapps documentation} is a good place to start.

\begin{description}
	\item[DateTimeBoxFactory] Defines a form control for the component \lstinline|DateTimeInput|, which is identified by the keyword \lstinline|datetimebox|. The widget shows a view element that displays the time and the date of a \lstinline|datetime| value.
	\item[GridPanelFactory] Defines a form control for the component \lstinline|GridLayoutPane|, which is identified by the keywords \lstinline|md2gridpanel| and \lstinline|gridpanel|. The widget enables to structure multiple view elements in a grid.
	\item[ImageFactory] Defines a form control for the component \lstinline|Image|, which is identified by the keyword \lstinline|image|. The widget is able to display a static image within your app.
	\item[SpacerFactory] Defines a form control for the component \lstinline|Spacer|, which is identified by the keyword \lstinline|spacer|. A spacer sets whitespace between components or within the grid of a \lstinline|GridLayoutPane|.
	\item[StackContainerFactory] Defines a form control for the component \lstinline|AlternativesPane|, which is identified by the keyword \lstinline|stackcontainer|. This widget encapsulates the stack container within \href{http://dojotoolkit.org/reference-guide/1.10/dijit/layout/StackContainer.html}{\lstinline|dijit/ layout/StackContainer|}. It provides a view element which has multiple views, but shows only one, similar to a book or a slide show. The user can navigate between them using specific keys. 
	\item[TextOutputFactory] Defines a form control for the component \lstinline|Label|, which is identified by the keyword \lstinline|textoutput|. This widget enables to display non-editable text.
	\item[TooltipFactory] Defines a form control for the component \lstinline|Tooltip|, which is identified by the keyword \lstinline|tooltipicon|. This widget offers a tooltip behind a question mark icon.
	\item[UploadImageOutputFactory] Defines a form control for the component \lstinline|UploadedImageOutput|, identified by the keyword \lstinline|uploadimageoutput|. The widget is able to display an image within your app, which is uploaded/specified by the user. 
\end{description}

Special dataform elements enable the use of uploaded files. The \lstinline|UploadedImageOutput| displays images which have been uploaded by an app's user using a \lstinline|FileUpload| input element. Given that both elements are mapped to an entity's attribute of type \lstinline|file|, these elements retrieve an image from, or store an image on the server, respectively. For this procedure, a specialised \lstinline|remoteConnection| needs to be defined by the modeller, thus defining the remote location of this service and a local path where this service is able to store files (\lstinline|fileUploadConnection|). Consequently, this remote connection can be different from every other remote connection used, \eg by content providers.

\begin{figure}[b!]
\centering % trim=l b r t
\includegraphics[clip,  trim=0 17.4cm 3.cm 0, scale=0.8]{Fig/upload-sequence.pdf}
\caption{Procedure of uploading and retrieving user-uploaded images}
\label{fig:remoteFileUpload}
\end{figure}

As a further consequence, the uploaded file is not directly stored in the database. Instead, when called by a \lstinline|FileUpload| element, the upload service stores the file on disk at the specified path (\lstinline|storagePath|) and returns an identifier string of this file to the calling client (cf. \Cref{fig:remoteFileUpload}, step 1).
This identifier is then used throughout the model, particularly as the value of a corresponding attribute of type \lstinline|file|  (cf. \Cref{fig:remoteFileUpload}, step 2).

When such an identifier is encountered as the value of an \lstinline|UploadedImageOutput|, this element calls a servlet at the \lstinline|fileUploadConnection|, passing the identifier as a parameter (cf. \Cref{fig:remoteFileUpload}, step 4). That way, the image is downloaded for the client just as soon as it is needed, instead of loading it at every initialisation of entities by a content provider. Note that currently only JPEG images should be uploaded, since the servlet always tries to output any given file using the content type \lstinline|image/jpeg|.



\subsubsection{List of Open Issues}
\label{sec:listOfOpenIssues}

The \lstinline!md2_list_of_open_issues! comprises all code necessary to display the list of open issues within the app. This list shows all workflow instances, whose state is at a workflow element that belongs to the current app. Currently, the data listed in this widget are the the workflow element name, the last fired event and a timestamp for the last modification of the workflow state. 
The list is included as \lstinline!dijit_Widget! and is listed as a \lstinline!Tool! in the app.json under the bundle specifications of the \lstinline!toolset!. In the \lstinline!ListOfOpenIssues!\lstinline!Controller! a \lstinline!DataView! is created, which uses the workflow store as a \lstinline!DataViewModel!. The workflow store is described in \Cref{workflow_store}.
Workflow instances are not only listed, but it is possible to start the workflow element by clicking on the respective entry. Then, the \lstinline!ListOfOpenIssuesController! handles the event \lstinline!onClicked! and calls the function \lstinline!startWorkflow! of the respective \lstinline!MD2MainWidget!. The workflow instance ID is retrieved in combination with the content provider IDs of its current state. With these, the content provider referenced in the \lstinline!workflowStateHandler! are set to their respective values.

\subsubsection{Local Store} \label{local_store}



The local store within the bundle \lstinline|md2_local_store| is one of three stores used in the context of \mapapps within \MD. This store implements some of the guidelines from the \lstinline|dojo/Store| interface, which means, that it offers the methods \lstinline|query|, \lstinline|get|, \lstinline|put|, \lstinline|add|, and \lstinline|remove|.
The local store can be used by a content provider (set to local within the controller model). This store saves all data as cookies in the browser. Thus, the store is not meant for consistent data storage.

\subsubsection{Location Service}
The \lstinline|LocationStoreFactory| provides methods to convert a longitude and latitude pair into an address (i.e. a country, city, street, postal) and vice versa. In the first case, the method \lstinline|_getAddressForLocation| is used, which takes two parameters for the longitude and latitude value. In the second case, the method \lstinline|_getLocationForAddress| is used. This method takes a single string as input, which contains all the address information (\eg \textit{Schlossplatz M\"unster 48149}). For both methods, the result is a JavaScript object.

ArcGIS is the underlying API that is used for this (reverse-) geocoding. The URL to use this service is specified in the \lstinline|manifest.json| of that bundle. Currently, the URL is:

 \lstinline|http://geocode.arcgis.com/arcgis/rest/services/World/GeocodeServer|. 

\subsubsection{Runtime}

This bundle contains the main logic of the \MD \mapapps framework, which is mainly based on the \lstinline|MD2MainWidget| object. Most other sub bundles just enhance the functions of this widget.

\paragraph{MD2MainWidget}
The \lstinline!MD2MainWidget! is for example responsible for the opening and closing of views. 
Each workflow element (see \Cref{sec:developAndDeployMultiApps}) has its own instance of a \MD main widget. This is specified in the respective controller of the workflow element bundle inside the app. That is, the \lstinline!manifest.json! of the workflow element bundle references an \lstinline!_md2AppWidget! for its controller. Once the controller is activated (\ie the \lstinline!activate! function is called), the respective \MD main widget instance is built. This \MD main widget is implemented in the file \lstinline!MD2MainWidget!, which serves as the basic starting point to start a workflow. Thus, it provides methods to start a workflow element. There are different ways a workflow can be started:
\begin{itemize}
	\item A workflow can be started directly from the \mapapps tool bar. Each workflow element marked as \lstinline|startable| in the model gets its own tool. After clicking on it, the method \lstinline|startWorkflowFrom|\lstinline|Tool| is started, which first resets all current workflow elements and the workflow handlers and then calls the \lstinline|startWorkflow| function.
	\item A workflow can also be started from the list of open issues (see section \ref{sec:listOfOpenIssues}). Therefore, the \lstinline|startWorkflow| is directly started from the \lstinline|OpenIssuesListController|.
\end{itemize}
When the workflow element window is closed and the same operation to start a workflow element is called again (reopening the same entry of the list of open issues or restart the same tool) the same window with all data entered before will be opened.
However, all changes are dropped if another workflow instance is opened in between!
Each \lstinline|MD2MainWidget| contains a runtime variable \$, which contains important objects needed within the context of the widget. While many objects are created anew for each widget, some objects such as the \lstinline|WorkflowStateHandler| (cf. \Cref{par:workflowStateHandler}) are globally used and thus should only be created once. Therefore, many of these objects are injected as singletons within the \lstinline|mainfest.json| of the bundle \lstinline|md2_runtime|.
\Cref{fig:InitMD2MainWidget} depicts this initialization process.

\begin{figure}
\centering
\begin{tikzpicture}[
	redbox/.style = {rectangle, fill=ercisred, text =white, draw=none, text centered, drop shadow},
	arrowlbl/.style = {pos=.5, black},
	>=stealth
]

\draw (-3.5,-2.25) rectangle (1.2,-0.5);
\draw [fill=gray!20!white] (-3.5,-0.95) rectangle (-0.7,-0.5) node[pos=.5] {\small{List of open issues}};
\draw [redbox] (-3.3,-2) rectangle (1,-1) node[pos=.5] {OpenIssueListController};

\draw [->, thick, ercisred] (1,-1.1) -- (4.6,-1.1) node[above, arrowlbl] {\lstinline!startWorkflow!} -- (4.6,0.4) -- (6.5,0.4);

\draw [->, thick, ercisred] (1,-1.9) -- node[above, arrowlbl] {\lstinline!getMD2MainWidget!} (6,-1.9);

\node[draw] (tool) at (-1.25, -3) {\small{\textit{map.apps tool}}};
\draw [->, thick, ercisred] (tool) -- (5.35,-3) node[above, arrowlbl] {\lstinline!startWorkflowFromTool!} -- (5.35,0.2) -- (6.5,0.2);

\draw [->, thick, ercisred] (-2.9,1) -- node[above, arrowlbl] {\lstinline!load!} (-1.9,1);

\node [above] at (-0.9,0.5) {\lstinline!activate!};
\draw [->, thick, ercisred] (-1.8,1) -- (-1.8,0.5) -- (0,0.5);
\draw [fill=gray!20!white](-1.9,1.1) rectangle (-0.8,1.5) node[pos=.5] {\small{WfE}};
\draw (-1.9,-0.25) rectangle (2.25,1.5);
\draw [redbox] (0,0) rectangle (2,1) node[pos=.5] {Controller};

\draw [->, thick, ercisred] (2,0.65) -- node[above, arrowlbl] {\lstinline!build!} (6.5,0.65);

\draw [fill=gray!20!white] (5.7,1.1) rectangle (8,1.5) node[pos=.5] {\small{MD2Runtime}};
\draw (5.7,-2.25) rectangle (11.25,1.5);
\draw [redbox] (6.5,0) rectangle (11,1) node[pos=.5] {MD2MainWidget};
%\node [rotate=270] at (10.7,0.35) {\small{startWorkflow}};
%\draw [->, thick, ercisred] (10,0.8) -- (10.5,0.8) -- (10.5,0.2) -- (10,0.2);

\draw [->, thick, ercisred] (7,0) -- node[right, arrowlbl] {\lstinline!register!} (7,-1);

\draw [redbox] (6,-2) rectangle (11,-1) node[pos=.5] {WorkflowStateHandler};

\end{tikzpicture}
\caption{Initialization of the \MD main widget in order to start a workflow}
\label{fig:InitMD2MainWidget}
\end{figure}

\paragraph{Actions}
Each action type defined in the \MD DSL must also exist in the \MD runtime bundle. Individual actions are stored in the subfolder \lstinline!simpleactions!. Moreover, an \lstinline!ActionFactory! must provide a method that returns an instance of the respective action (\eg \lstinline!getLocationAction! for the \lstinline!LocationAction!). All actions provide a method \lstinline!execute! that implements the action. An individual constructor allows to initialize the action, \eg setting a city's name for a \lstinline!LocationAction!. The \lstinline!Action!\lstinline!Factory! is instantiated in the method \lstinline!build! of the \lstinline!MD2MainWidget! (cf. \Cref{fig:InitMD2MainWidget}). Thus, every workflow element can access and use this factory.

\paragraph{Content Provider}\label{par:contentProvider}
The content providers are responsible for saving and persisting the state of one or multiple objects. While the generated code only provides factories for the creation of content providers, the actual code is located in the static bundles. Each content provider is instantiated with a unique name, the app ID it belongs to, a store (either remote or local), the information whether it is a provider for a list of objects, a filter restricting the queried items and the information whether it is a remote or a local store.
Besides functions to get or set the content of the provider, functions are offered to access the injected store and persist the data.
Additionally, each content provider can inform other components of the app about changes within attributes. This can be used for example to refresh the values of view elements.
The function \lstinline|restore| and \lstinline|reset| are used within the \lstinline|WorkflowStateHandler| to  influence the state of the content provider, for example when the workflow instance is changed.

\paragraph{Data Mapper}
Whenever a value within a content provider changes it is necessary to inform the view elements to be able to refresh them. This is one within the classes of this folder.

\paragraph{Data Types}
In this folder all data types known by the \MD \mapapps framework are listed. Each data type provides additional functions for working with the data objects such as cast or compare operations. The \lstinline|TypeFactory| is used to instantiate an object according to its data type and is injected to the runtime variable \$ within the \lstinline|MD2MainWidget|.

\paragraph{Entities}\label{par:Entities}
This folder contains all internal entities known to the \MD \mapapps framework. Currently, this is the \lstinline|Location| entity and the abstract class \lstinline|_Entity|, which is inherited by all generated entities.

\paragraph{Events}
In the \MD modelling language it is possible to define actions based on events. Examples for such events are changes or clicks on view elements. To be able to map this behaviour in \mapapps, each possible events has its own class which subscribes a topic associated with the type of event it represents. The \lstinline|EventRegistry| has a list of all possible events and the root event classes enable to register actions to the specific events.

\paragraph{Handler}
This folder contains handlers for global events. These mainly display results in info or warning messages. Certain actions are bound to events by subscribing to topics. One example is the data action bound to the topic \lstinline[language=Javascript]|"md2/contentProvider/dataAction/${appId}"|. If any component is publishing a status (\lstinline[language=Javascript]|"success"| or \lstinline[language=Javascript]|"error"|) for an action (\eg \lstinline[language=Javascript]|"load"| or \lstinline[language=Javascript]|"save"|) a respective info message is shown in the lower right corner of the application.

\paragraph{Resources}
This folder contains images and style files.

\paragraph{Templates}
This folder contains the root html file of the \lstinline|MD2MainWidget|.

\paragraph{Validators}
In the model validators can be defined for view elements. This folder contains all existing validators, which can be created using the \lstinline|ValidatorFactory| which is injected in the runtime variable \$.

\paragraph{View}
The \lstinline|MD2MainWidget| is responsible for creating the views for the workflow elements. For this purpose it uses a \lstinline|ViewManager| which creates the view elements based on the \lstinline|view| entries within the manifest.json of the respective workflow element. The type of a view element is therefore linked to a dataform component either contained in the bundle \lstinline|md2_formcontrols| or within the external conterra bundle \lstinline|base/dataform|.

\paragraph{Workflow}\label{par:workflowStateHandler}
The \lstinline|WorkflowStateHandler| and the generated \lstinline|WorkflowEventHandler| are responsible together for managing the state of the current workflow instance. This includes the transitions between workflow elements within the context of one app as well as firing workflow events to the backend.  
For each started workflow instance a unique ID is generated and assigned to that instance. This is done in the method \lstinline!startWorfklow! of the \lstinline!MD2MainWidget!. A \lstinline!WorkflowStateHandler! provides methods to set and get the currently active workflow instance ID in a global context. This information is needed to suspend a workflow and to resume that workflow at a later time. The variable \lstinline|_lastStartedTool| provides information about how the workflow instance has been started. This is important when it comes to decide for a new \lstinline!startWorfklow! evaluation, if the current workflow instance has to be resumed or a new one should be created.

To change the current workflow element, the \lstinline|WorkflowStateHandler| provides the methods \lstinline|change|\lstinline|WorkflowElement| and \lstinline|fireEventToBackend|. 
The first one opens another \lstinline|MD2MainWidget|, as the workflow is continued in the same app.
The latter one will exit the workflow instance for the current app and close all windows, as the workflow is supposed to continue in another app. Additionally, the backend is informed about this step by calling the \lstinline|fireEventToBackend| method of the \lstinline|WorkflowStore| (cf. \Cref{workflow_store}). Since the \lstinline|WorkflowStore| is supposed to send the content provider IDs, this leads to a problem. 
The saving operation of the content providers is usually done shortly before the workflow event is fired. Since the content provider IDs are not set until the backend has answered to the web service calls, they may not be accessible yet. For this purpose, the \lstinline|WorkflowStateTransaction| is used. 
With each new workflow instance ID a new transaction is created. It keeps a list of all started content provider saving operations and only allows to fire a workflow event when no save operations are in progress. Additionally, the transaction is informed via a subscription about the termination of each content provider operation and will then retry to fire a workflow event, if one was queued before.

\subsubsection{Store}\label{store}

The \lstinline|md2_store| bundle provides the second of the three stores. It could also be called remote store, since it provides access to external data storage. It is again an implementation of the \lstinline|dojo/Store| and implements all necessary functions. The store is used within a content provider to query the current state of the objects belonging to the current workflow instance. In contrast to the local store, data which is saved within this store is persisted throughout the whole application landscape. For this purpose each store needs to be provided with an URL, pointing to the respective backend server. The data is then queried and stored using REST web services. One implementation of such a backend is automatically generated by the backend generator described in \Cref{sec:backend}. For the store to be able to find its respective backend, the URLs need to match each other (see \Cref{subsubsec:Backend}). 

\subsubsection{Workflow Store} \label{workflow_store}

The workflow store within \lstinline|md2_workflow_store| differs from the other stores provided by the static bundles. While the local (cf. \Cref{local_store}) the remote store (cf. \Cref{store}) are used within a content provider, the workflow store has its own purpose. It is used to save and query the status of all workflow instances. It is injected in the \lstinline|WorkflowStateTransaction| of the \lstinline|md2_runtime| bundle (cf. \Cref{par:workflowStateHandler}). There, it is responsible of informing the backend of any changes within the workflow instance. In the model this action can be represented by the \lstinline|FireEventAction|. Beside the information which event has been fired and which was the current workflow element and instance ID, the backend gets a list of all IDs registered in the content providers. This is necessary in order to be able to restore the state of the content providers in another app.

The workflow store is also referenced in the list of open issues. The workflow store implements, similar to the other stores, the functions of the \lstinline|dojo/Store|. This makes it possible to hand the workflow store over to a \lstinline|DataView|, which then displays the information retrieved from the store. The \lstinline|query| function enables to query the whole list of the list of open issues.

For the workflow store to work correctly, the app.json needs to provide an appropriate configuration for the REST location and the current app ID, for which the web service should filter. The following snippet (with equivalent values) is automatically generated:
\begin{lstlisting}[language=Javascript]
"md2_workflow_store": {
	"MD2WorkflowStore": {
		"uri": "http://localhost:8080/ReferenceProject.backend/service/workflowState/",
		"app": "Citizenapp"
	}
}
\end{lstlisting}

% end of section for static map.apps implementation

\subsection{Generated \mapapps Implementation} 

The generated code of the \mapapps applications contains everything, that can not be generalised in the static implementation. This includes mainly the defined content providers, entities, and workflow elements, as well as the workflow mechanism described in the workflow model. Additionally, some components of the static \mapapps implementation are injected with model specific settings.

It is important to note that each app definition within the workflow model will result in its own app. Each app gets its own content provider, model, and workflow element implementation. However, files for those objects are only created when they are necessary within the context of the specific app.

\subsubsection{Content Providers}
The content providers are created as an individual bundle within each app. Each content provider is represented within its own file. However, the file does not incorporate the content provider itself, but merely a content provider factory. It is used to create an instance of the content provider class included in the static \lstinline|md2_runtime| bundle (cf. \Cref{par:contentProvider}).

Each content provider is created using the name defined in the model, the app ID and an instance of the respective local or remote store. The stores are created using a store factory injected within the manifest.json.
Additionally, the remote stores are initialised with the URL of their corresponding remote connection.

Besides the content providers defined within the model, two local content providers are created (namely \lstinline|__returnStepStackProvider| and \lstinline|__processChainControllerStateProvider|). They enable the usage of process chains within a workflow element. Therefore, they save the steps taken and their respective state so that it is possible to return to a previous step.

\subsubsection{Models}

The entities are grouped within the bundle \lstinline|md2_models|. Each defined entity and enum in the model gets its own file. Additionally, the two entities \lstinline|__ProcessChainControllerState| and \lstinline|__returnStepStack| are created to map the state of the two content providers listed above. Each entity inherits from the static class \lstinline|_Entity| described in \Cref{par:Entities}. They describe their own data type as well as the types of all their attributes, or, in case of an enum, the possible values. Additionally, the entities have an \lstinline|_initialize| function to create empty attribute types for all attributes, except for the referenced entities. For those the values have to be set manually using a return value of another content provider later in the code.

\subsubsection{Workflow Elements}

For each workflow element of the app an individual bundle is created. This bundle specifies a \lstinline|Controller|, which is the instance factory for a \lstinline|MD2MainWidget|. The \lstinline|MD2MainWidget| itself is further specified within the manifest.json. Here, all view elements are defined in addition to the app ID, the workflow element ID, the webserviceBackendURI, the window title, and the action called upon initialization of the workflow element. This \lstinline|onInitialized| action is executed by the \lstinline|MD2MainWidget| when opening the widget. Besides the \lstinline|Controller|, all actions which inherit from the static class \lstinline|_Action| are specified. The \lstinline|CustomActions.js| implements a class that serves as an instance factory for all actions defined within the workflow element.

\subsubsection{Workflow}
This bundle contains the \lstinline|WorkflowEventHandler|, which keeps a list of all  \lstinline|MD2MainWidget|. Additionally, it contains the specification for which workflow follows which, and for the differentiation between the case when an event has to be fired to the backend or a workflow element must be changed within the app. After exiting the context of a workflow instance, the event handler has the capability to reset all \lstinline|MD2MainWidget| instances registered in order to be prepared for a new instance.

\subsubsection{App.json} \label{subsubsec:mapapps-generated-appjson}
The app.json is used to inject further information into the static bundles. These settings are mainly backend URLs and app IDs, which are needed in this components.
Besides that, the app.json  contains a list of all bundles used within the app context.


\section{Backend}\label{sec:backend}
% !TEX root = ../md2-user-handbook.tex
% !TeX spellcheck = en_GB
% !TeX program = xelatex

\lstset{language=Simple}

The \MD backend is implemented in Java. It is automatically generated by from the \MD model. Note, that some parts of the backend are static while others are contingent upon the model.

\subsection{Beans}
\label{subsec:beans}
For entities that are used in at least one remote content provider a stateless session bean is generated. Such a bean provides basic methods to create or manipulate entities of its type. The Java Persistence API (JPA) is used for the persistent storage of the data. The persistence configuration file is located at \lstinline|META-INF/persistence.xml|. Currently, EclipseLink is used as persistence service.

Additionally, a static session bean is generated for the workflow state of the instances of a workflow. Internally, a workflow instance is identified by an unique ID that is represented as an integer. However, since a workflow instance is generated at the client side, a single client cannot assure that a specific number is not used by another workflow instance of another client. Therefore, every client generates its own \lstinline|instanceId| that is a hash value of the current time and other variables, and thus supposed to be unique. This generated hash value is represented as a string value.

\subsection{Datatypes}
The datatypes used in the backend are static and implemented as simple wrappers. For instance, every entity has a unique internal identification number (\lstinline|internalID|) that is an integer value. The respective wrapper implementation is depicted in the following listing. 

\begin{lstlisting}[language=Java, label=lst:intIdWrapper, caption=An integer wrapper for the internal identification number]
@XmlRootElement(name = "internalId")
public class InternalIdWrapper {
	
	@XmlElement
	protected int __internalId;
	
	protected InternalIdWrapper() {
		// no-arg default constructor necessary
	}
	
	public InternalIdWrapper(int integer) {
		this.__internalId = integer;
	}
}
\end{lstlisting}

\subsection{Entities}
The entities and enumerations defined in the \MD model are generated into the subpackage \lstinline|entities.models|. Moreover, two static Java classes are generated:  \lstinline|RequestDTO| and \lstinline|WorkflowState|. The former is used as an encapsulation for all client requests, \eg to create a corresponding REST request. The latter is used as a representation of the state a particular workflow instance has. A workflow can consist of multiple workflow elements, that in turn can fire different events. Thus, every started workflow (\ie every workflow instance) must keep track of its current workflow element and the last event fired in it.

\subsection{File Download Servlet}
This servlet is used to deliver uploaded files to a requesting client. It is accessible at the path \lstinline|/DownloadFile| below the web root of your deployed project. 

In a \lstinline[language=Simple]|GET| request to this servlet you need to set the parameter \lstinline|file| to the identifier of an uploaded file, which was returned by the file upload web service before (cf. \Cref{fig:remoteFileUpload}). 
All files are stored in the file system. Therefore, the download servlet needs to look for files using the identifier in a central directory, which is also referenced by the upload web service. This location  is defined in the generated \lstinline[language=Java]|Config.UPLOAD_FILE_STORAGE_PATH|, which is derived from the \lstinline|storagePath| element in the file upload remote connection in your model.

Note, that currently only images can be delivered as the download servlet assumes the content type to be \lstinline|image/jpeg|. This could be changed in the future by storing the correct content type during upload and retrieving it in this servlet.

\subsection{Web Services}
Similar to the generation of the stateless session beans, a web service is only generated for those entities that are used in at least one remote content provider. Those web services provide simple access to entity data.
Additionally, some static web service are generated that are used for specific features. Those are explained in the following.

\subsubsection{Calls to External Webservices}
As a simple way to interact with external services, a web service \lstinline|CallExternalWebServiceWS| in the backend allows to call another web service, that might be on a different system or server. The web service in the backend provides a method that takes a JSON-encoded object as an input. This object must contain the URL, the REST method type and the set of parameters of that method. For example, the following listing depicts how such a JSON object is constructed in map.apps.

\todo{Add workaround for HashMap @CARO, @JAN, @MALTE}

\begin{lstlisting}[language=Javascript, label=lst:callExtWSJSON, caption=JSON-encoded object containing information to call an external web service]
data: json.stringify({
  "url": this._url,
  "requestMethod": this._method,
  "queryParams": this._queryParams,
  "body": this._bodyParams
})
\end{lstlisting}

\subsubsection{Offer Webservices to Start Workflow}
Besides the possibility to start a workflow through an app it is possible to invoke it using a webservice. The description of the corresponding model language is described in \Cref{subsec: WorkflowControlThroughWS}. For each invokable workflow element a webservice is created and for each invoke definition a webservice endpoint is specified, including the defined parameters and the creation of the required entities. After the entities are saved using the internal beans, a \lstinline|workflowState| is persisted using the workflow element the webservice belongs to. Additionally, the \lstinline|lastEventFired| is set to the defined text specified in the workflow model after the \lstinline[language=MD2]|invokable| keyword, or to a default if not specified. The entity IDs returned by the internal beans are then injected as the content provider IDs. Directly afterwards, the workflow instance is accessible within the list of open issues of all app that are allowed to view the invoked workflow element. Since a new workflow instance is created, the backend is creating a new random UUID for each webservice call.

For each endpoint a method \lstinline|@POST| or \lstinline|@PUT| can be defined. The used parameter types are \lstinline|@FormParam|. The path which has to be used to call the webservice endpoint consists of the workflow element name and the specified path in the invoke definition.

\subsubsection{Event Handler} 

For the communication across apps, the backend offers an event handler web service. This web service handles all workflow events that are fired in one app and need to start a workflow element in another app. Required parameters for this web service are

\begin{itemize}
\item the instance ID of the workflow instance,
\item the event which was fired,
\item the content provider IDs,
\item and the current workflow element which fired the event.
\end{itemize}

The event handler web service uses these parameters to perform adjustments in the workflow state of the current workflow instance. This includes setting the last event fired and the current workflow element. Furthermore, the content provider IDs are stored in the workflow state so that subsequent apps can load data from content providers using these IDs.

\subsubsection{Workflow State} 

The workflow state web service allows to retrieve open workflow instances or add new ones. Whenever the list of open issues is opened in an app, this app sends its name to the workflow state web service. The web service then returns all workflow states whose current workflow element is part of the app with the given app name. For this purpose, the belongingness of workflow elements to apps is originally derived from the DSL model and stored in a hashmap in the backend.

Furthermore, for every new workflow instance, a new workflow state needs to be created. To do so, the workflow state web service is called as soon as the app which started the workflow hands the control over to the backend. This app generates a globally unique identifier (as described in \Cref{subsec:beans}) and provides it in the web service call.

\subsubsection{File Upload}

As another web service, a REST endpoint for uploading files is provided.
In contrast to the other web services, it expects an input format of \lstinline|MULTIPART_FORM_DATA|, thus allowing image uploads from HTML forms.

Given an uploaded file, it creates a file with a unique file name using the \lstinline[language=Java]|File.createTempFile()| interface. The file is stored in the location specified in \lstinline[language=Java]|Config.UPLOAD_FILE_STORAGE_PATH| (or \lstinline|storagePath|) and the generated file name is returned to the invoking client. No further information about the file is stored or checked, \ie original file name and content type are lost.

\subsubsection{Version Negotiation} 
This web service can be used by generated apps to check whether they were generated from the same model version as the backend. Consequently, this is only useful if the modeller updates the model version after making changes to the data model.


\section{map.apps Generator}
% !TeX spellcheck = en_GB
% !TeX program = xelatex
% !TeX root = ../md2-user-handbook.tex

\todo{empty @ALL [why is this chapter after the map.apps Implementation?]}



%%%%%%%%%%%%%%%%%%%%%%%%%%%%%%%%%%%%%%%%%%%%%%%%%%%%%%%%%%%%%%%%
%% appendix

\clearpage
\appendix
\pagenumbering{Roman}		%% roman page numbers for the appendix
%\input{90-Appendix}

\chapter{Sample Workflow}
\input{Appendix/A1-SampleWorkflow}

\chapter{Known Issues and Suggestions for Future Development}
% !TeX spellcheck = en_GB
% !TeX program = xelatex
% !TeX root = md2-user-handbook.tex

Since the generators for iOS and Android were not further development, they need to be adapted in the future in order to be compatible with the new version of the DSL and support all new features. Special attetion should be paid to the former workflows which were renamed as process chains during the course of the project seminar. In the iOS and Android implementations, these naming adjustments also need to be performed.

Furthermore, the division into workflow elements allows for a very modular architecture of the created apps. To exploit this advantage, a renaming of workflow elements should be allowed in the model in a way that different apps can use the same workflow element using different names for them. This way, it would be possible to determine very precisely which workflow element in which app is to be started rather than starting a workflow element which can be processed by any app that has the respective workflow element assigned.

Another possible extension is to allow return values when external webservices are called. This would enable programmers to include almost arbitrary functionality in the model without the need to offer it as explicit construct in the DSL. The only requirement is that return values comply with the data types supported by the \MD framework to ensure that they can be used in a purposeful way.

In addition, there are still improvement possibilities in the DSL using validators. One possibility is to check whether the fire event action in a workflow element is the last action and warn if it is not. This can be helpful, for example, when a save action follows a fire event action. In this case, saving data will not be performed, since the fire event action immediately forwards control to the next workflow element. Another possibility is to throw warnings or errors when a controller maps something to view elements which are never used by the controller, e.g. mapping something to a view which only appears in a different app. Finally, in the case that several content providers are used, a validator can check whether all content providers are saved, in order to ensure that a modeller does not forget any save actions.

Other possible features are

\begin{itemize}
\item call other external applications apart from REST,
\item extend the location features (rather map.apps specific):
\subitem display locations on a map,
\subitem convert a click on a map into location data,
\subitem convert coordinates into an address,
\item allow the definition of custom icons for startable elements,
\item support white spaces in project names,
\item support temporary offline usage,
\item generate only relevant content for each app, e.g. only generate entities which are used by the apps,
\item provide build scripts for \MD,
\item provide auto formatting in the IDE,
\item access foreign apps such as the phone, camera and GPS.
\item customizable columns within the list of open issues.
\end{itemize}




\chapter{Further Improvements to your Development Environment}
% !TeX spellcheck = en_GB
% !TeX program = xelatex
% !TeX root = md2-user-handbook.tex

\section{Reference a Generated App in the Development Project}
\label{subsec:link-apps}

By using a symlink, a running NetBeans instance will automatically notice changes to the generated app. Consequently, if the Jetty server is running, the newly generated app will automatically be published and made available in the browser.

\begin{enumerate}
\item Open a terminal and navigate to the \mapapps NetBeans project directory (e.g. the extracted \lstinline[language=Simple]|sample ProjRemote| from step \ref{item:extraction} in \Cref{subsec:basic-setup}).
\item Navigate to the \lstinline[language=Simple]|apps/| directory using \lstinline[language=Simple]|cd src/main/js/apps|.
\item Create a symbolic link using an appropriate command (where \lstinline[language=Simple]|<PROJECT_NAME>| is the name of your \MD Project in Eclipse, \lstinline[language=Simple]|<ECLIPSE_PROJECT_LOCATION>| is its location, and \lstinline[language=Simple]|<APP_NAME>| is the name of the generated app(s)): \label{item:create-symlink}
	\begin{enumerate}
	\item Windows:\\
	\lstinline|mklink /j <APP_NAME> <ECLIPSE_PROJECT_LOCATION>\<PROJECT_NAME>\src-gen\ <PROJECT_NAME>.mapapps\<APP_NAME>|
	
	\item Linux / OS X: \\
	\lstinline|ln -s <ECLIPSE_PROJECT_LOCATION>/<PROJECT_NAME>/src-gen/<PROJECT_NAME>. mapapps/<APP_NAME> <APP_NAME>|
	\end{enumerate}
	
\item Repeat step \ref{item:create-symlink} for every generated app that you would like to have refreshed automatically.
\end{enumerate}


\section{Jetty: Allow Serving of Symbolically Linked Files}
\label{subsec:jetty-symbolic-files}

On Linux/OSX, Jetty by default does not serve symbolically linked files due to security concerns.
To override this setting (which is not recommended in a production environment), put the provided file \lstinline[language=Simple]|jetty-web.xml| into the folder \lstinline[language=Simple]|/src/test/webapp/WEB-INF/| of your map.apps project.




\chapter{Backend Connection Specification} \label{chp:appendix:backend-ws-spec}
% !TeX spellcheck = en_GB
% !TeX program = xelatex
% !TeX root = md2-user-handbook.tex

\todo{@ALL review pls}
\section{Resource paths}
Format: \\
VERB - Path - Request body \hfill \lstinline|<Status> - <Response body>|

\subsection*{Entities}

Load\\
\lstinline|GET - /<entity.name>/?filter=<filter>| \hfill \lstinline|200 OK - List<Entity>| \\
\lstinline|GET - /<entity.name>/first?filter=<filter>| \hfill \lstinline|200 OK - Entity or 404 NOT FOUND|
\\
\\
Save\\
\lstinline|PUT - /<entity.name>/ - List<Entity>| \hfill \lstinline|200 OK - List<{ “__internalId”: <id> }>|
\\
\\
Delete\\
\lstinline|DEL - /<entity.name>/<id>| \hfill \lstinline|200 OK or 404 NOT FOUND|


\subsection*{Remote validations}
\lstinline|GET - /md2_validator/<remoteValidator.name>/ - Entity| 

~ \hfill \lstinline|200 OK - ValidationResult object|
\\ \\
\lstinline|GET - /md2_validator/<remoteValidator.name>/?attrName1=content&attrName2=content ... &attrNameN=content| 

~ \hfill \lstinline|200 OK - ValidationResult object|
\\
\\
attrNameX is a fully qualified name, having \\
\lstinline|contentProviderName.path.to.attribute|


\subsection*{Filter parameter}
\lstinline!not <Attribute> (equals|greater|smaller|<=|>=) (<Int>|<Float>|<String>|<InputField>) !\\
\lstinline!((and|or)(not)? <Attribute> (equals|greater|smaller|<=|>=) (<Int>|<Float>|<String>|!\\
\lstinline!<InputField>))*!

\subsection*{Resource for model version checks}
The model version should be checked by the apps for all remote connections. Requests are only valid if the server accepts the current model version. \\
\lstinline|GET /md2_model_version/current| \hfill \lstinline|200 OK - <version>| \\
\\
\lstinline|GET /md2_model_version/is_valid?version=<version>| 

~ \hfill \lstinline!200 OK - { "isValid": (true|false) }!


\section{JSON format conventions}
List<Entity>:
\begin{lstlisting}[language=Javascript]
{
	"entityName": [
	{
		"attribute": <Value type see below>,
		[...]
	},
	{
		"attribut": <Value type see below>,
		[...]
	} [...]
	]
}
\end{lstlisting}
Having <Entity> = Entity without root node
\\
\\
Entity:
\begin{lstlisting}[language=Javascript]
{
	"entityName": [
		"attribute": <Value type see below>,
		[...]
	}
}
\end{lstlisting}
~
\\
Validation Result:
\begin{lstlisting}[language=Javascript]
{
	"ok": (true|false),
	"error": [
	{
		"message": "Allgemeine Fehlermeldung",
		"attributes": ["attribut1", "attribut2"]
	},
	{
		"message": "can’t be blank",
		"attributes": ["forename", "surname"]
	}
	]
	[...]
}
\end{lstlisting}
~
\\ \\
Mapping for data types (language data type -> JSON type for attribute values) \\
\lstinline|Enum -> Int (index of the currently selected value)| \\
\lstinline|Int -> Number|	
\\ \\
\lstinline|Float -> Number| \\
\lstinline|<Everything else> -> String| \\
\lstinline|Date -> String im Format yyyy-mm-ddThh:mm:ss+hh:mm| \\


\section{Examples}

GET \lstinline|/customer/first| returning one customer
\begin{lstlisting}[language=Javascript]
{
	"customer": {
		"__internalId": "0",
		"firstName": "Ulrich",
		"lastName": "M\u00c3\u00bcller",
		"membership": "1",
		"professionalCategory": "0"
	}
}
\end{lstlisting}
~
\\
GET \lstinline|/customer| returning multiple customers
\begin{lstlisting}[language=Javascript]
{
	"customer": [
	{
		"__internalId": "0",
		"firstName": "Ulrich",
		"lastName": "M\u00c3\u00bcller",
		"membership": "1",
		"professionalCategory": "0"
	},
	{
		"__internalId": "0",
		"firstName": "Hans",
		"lastName": "Dampf",
		"membership": "1",
		"professionalCategory": "0"
	}
	]
}
\end{lstlisting}



%% Stop section "numbering" after appendix

\setcounter{secnumdepth}{0}

%%%%%%%%%%%%%%%%%%%%%%%%%%%%%%%%%%%%%%%%%%%%%%%%%%%%%%%%%%%%%%%%
%% lists of figures and tables

%\clearpage
%\listoffigures
%\listoftables

%%%%%%%%%%%%%%%%%%%%%%%%%%%%%%%%%%%%%%%%%%%%%%%%%%%%%%%%%%%%%%%%
%% Define Abbreviations
%\clearpage
%\section{List of Acronyms}
%\begin{acronym}
%\acro{AACS}{Advanced Access Content System}
%\acro{XOR}{exclusive or}
%\end{acronym}

%\clearpage
%%%%%%%%%%%%%%%%%%%%%%%%%%%%%%%%%%%%%%%%%%%%%%%%%%%%%%%%%%%%%%%%
%% bibliography (if needed)

%\nocite{*}
%\bibliographystyle{plain}
%\bibliography{lit}

%\printbibliography[heading=bibintoc] % biblatex
\end{document}