% !TeX spellcheck = en_US
% !TeX program = xelatex
% !TEX root = ../md2-user-handbook.tex

\paragraph{Main}
The \lstinline!main! object contains all basic information about a project. Each project must contain exactly one \lstinline!main! object that can be in an arbitrary controller.

\begin{lstlisting}
main {
	appVersion STRING
	modelVersion STRING
	workflowManager [RemoteConnection]
	defaultConnection [RemoteConnection]
	fileUploadConnection [RemoteConnection]
}
\end{lstlisting}

The attributes are explained as follows:
\begin{itemize}
\item \lstinline!appVersion! -- a string representation of the current app version, \eg \enquote{RC1}
\item  \lstinline!modelVersion! -- a string representation of the current model version that has to be in accordance with the model version of the backend
\item  \lstinline!defaultConnection! (optional) -- a default remote connection can be specified, so that it is not necessary to specify the same connection in each content provider
\item \lstinline!workflowManager! -- the remote connection to the workflow manager, which may be different from the default backend connection
\item \lstinline!fileUploadConnection! (optional) -- a remote connection to the file upload server must be specified if a file is expected to be uploaded or an uploaded file to be displayed. This connection can be similar to the default connection, but storing large amounts of files on a separate server is encouraged. Specifying the connection is obligatory if at least one \lstinline!UploadedImageOutput! or \lstinline!FileUpload! is present in the views.
\end{itemize}

Furthermore, the controller layer is subdivided into one or more workflow elements. While a workflow describes the interaction of workflow elements, their internal functionality needs to be specified in the controller layer. Each workflow element can be seen as an independent controller which is responsible for the successful execution of its functionality. In general, workflow elements are structured as follows

\begin{lstlisting}
WorkflowElement <workflowElementName>{
	defaultProcessChain <ProcessChain>
	onInit {
		<...>
	}
	<...>
}.
\end{lstlisting}

Workflow elements work with \lstinline!ProcessChains! that allow to define a sequence of steps inside the workflow element and are further described in \Cref{sec:processChain}. Since several process chains can be defined in a single workflow element, the \lstinline!defaultProcessChain! keyword is used to set a default process chain. This process chain will then be used as starting point for the workflow element and the first view to be shown will be derived from it.

The \lstinline!onInit! block is used to define everything that is supposed to happen upon the initialization of the workflow element, \eg binding actions to buttons or mapping content to view fields. In the \lstinline!onInit! block, no workflow events may be fired, as this means handing off control from the workflow element directly during the initialization. This is enforced by through a validator that raises an error in case the modeler tries to throw an event here.

Typically, workflow elements should be modeled in a way that they fire a workflow event after termination of their functionality to start a new workflow element or end the whole workflow. This can be done using the SimpleAction \lstinline!FireEventAction!.

\paragraph{Actions}
An action provides the user the possibility to declare a set of tasks. An action can be either a \lstinline!CustomAction! or a \lstinline!CombinedAction!.

A \textit{CustomAction} contains a list of CustomCodeFragments where each CustomCodeFragment contains one task.

\begin{lstlisting}
CustomAction <Action> {
	<CustomCodeFragment>
	<...>
}
\end{lstlisting}

For each type of task there exists a specific CustomCodeFragment that is distinguished by the keyword that introduces it.

\begin{itemize}
\item \lstinline!bind <Action1> ... <ActionN> on <Event1> ... <EventN>!
\item \lstinline!bind <Validator1> ... <ValidatorN> on <ViewElement1> ... <ViewelementN>!
\item \lstinline!unbind <Action1> ... <ActionN> from <Event1> ... <EventN>!
\item \lstinline!unbind <Validator1> ... <ValidatorN> from <ViewElement1> ... <ViewelementN>!
\item \lstinline!call <Action>!
\item \lstinline!map <ViewElement> to <ContentProviderField>!
\item \lstinline!unmap <ViewElement> from <ContentProviderField>!
\item \lstinline!set <ContentProvider> = <Expression>!
\item \lstinline!set <ViewElement> = <Expression>!
\item \lstinline!if (<Condition>) { <CustomCodeFragment> }!
\item \lstinline!elseif (<Condition>) { <CustomCodeFragment> }!
\item \lstinline!else {<CustomCodeFragment>}!
\end{itemize}

The main tasks are binding actions to events, binding validators to view elements and mapping view elements to model elements. For every task there is a counterpart for unbinding and unmapping. \lstinline!call! tasks can call other actions, \lstinline!set! operations set the value of a content provider field or a view element. The \lstinline!if!, \lstinline!elseif! and \lstinline!else! blocks allow to model case distictions, \eg based on user input.

Actions are bound to events. There are several types of actions and events available. CustomActions and CombinedActions are referenced externally whereas SimpleActions are declared directly. For events, there are local event types that listen to the state of a certain view element as well as global event types. The most powerful event type is the OnConditionEvent.

\textit{SimpleActions} provide a quick way to perform functionality:
\begin{itemize}
\item ProcessChainProceedAction: \lstinline!ProcessChainProceed!
\subitem proceed to the next ProcessChain step
\item ProcessChainReverseAction: \lstinline!ProcessChainReverse!
\subitem go back to the last ProcessChain step
\item ProcessChainGotoAction: \lstinline!ProcessChainGoto <processChainStep>!
\subitem change to the given ProcessChain step
\item SetProcessChainAction: \lstinline!SetProcessChain [ProcessChain]!
\subitem changes the current ProcessChain
\item GotoViewAction: \lstinline!GotoView (<ViewElement>)!
\subitem change to the given view element
\item DisableAction: \lstinline!Disable (<ViewElement>)!
\subitem disables a view element
\item EnableAction: \lstinline!Enable (<ViewElement>)!
\subitem enables a view element
\item DisplayMessageAction: \lstinline!DisplayMessage (<SimpleExpression>)!
\subitem displays a message
\item {ContentProviderOperationAction: \lstinline!ContentProviderOperation (<AllowedOperation>!

 \lstinline!<ContentProvider>)!}
\subitem perform a CRUD action (save, load, remove) on the given ContentProvider
\item ContentProviderResetAction: \lstinline!ContentProviderReset (<ContentProvider>)!
\subitem resets the given ContentProvider
\item FireEventAction: \lstinline!FireEvent (<WorkflowEvent>)!
\subitem fires a workflow event to the backend. In response a new workflow element will be started or the workflow terminated.

\item WebServiceCallAction: \lstinline!WebServiceCall (<WebServiceCall>)!
\subitem sends a request to call an external web service to the backend (for details cf. section \ref{subsec: CallingWebServices}).

\item The LocationAction allows to extract an address from content provider fields and generate a punctual location (\ie longitude and latitude) for this address. For this purpose, all input and output fields have to be defined. The calculation result for longitude and latitude will be written in their respective output fields. The LocationAction is structured as follows:

\begin{lstlisting}
Location (
	inputs (
		cityInput <ContentProviderPath>
		streetInput <ContentProviderPath>
		streetNumberInput <ContentProviderPath>
		postalInput <ContentProviderPath>
		countryInput <ContentProviderPath> )
	outputs (
		latitudeOutput <ContentProviderPath>
		longitudeOutput <ContentProviderPath>)
).
\end{lstlisting}

 
\end{itemize}

There are different event types available:
\begin{itemize}
\item \lstinline!ElementEventType! -- onTouch, onLeftSwipe, onRightSwipe, onWrongValidation; preceded by a dot and a reference to a ContainerElement or ContentElement
\item \lstinline!GlobalEventType! -- onConnectionLost
\item \lstinline!OnConditionEvent!
\end{itemize}


The \textit{OnConditionEvent} provides the user the possibility to define own events via conditions. The event is fired when the conditional expression evaluates to true.
\begin{lstlisting}
event NAME {
	<Condition>
}
\end{lstlisting}

A \textit{condition} allows to combine conditional expressions using the operators \lstinline!and!, \lstinline!or! and \lstinline!not!. Conditional expressions evaluate to true or false. They can be BooleanExpressions, EqualsExpressions or GUIElementStateExpressions that check the state of a ViewGUIElement. In addition, \MD supports mathematical expressions such as \lstinline!equals!, \lstinline!>!, \lstinline!<!, \lstinline!>=!, and \lstinline!<=!.

\textit{Validators} are bound to view elements. The validator can be a referenced element or a shorthand definition can be used in place.
\begin{lstlisting}
bind|unbind validator
	<[Validator]> <...>
	on|from
	<[ContainerElement] | [ContentElement]> <...>
\end{lstlisting}

The shorthand definition has the same options but does not allow reuse.
\begin{lstlisting}
bind|unbind validator
	<IsIntValidator | NotNullValidator | IsNumberValidator | IsDateValidator | RegExValidator |NumberRangeValidator | StringRangeValidator (<parameters>)
	on|from
	<[ContainerElement] | [ContentElement]> <...>
\end{lstlisting}

A detailed description for validator types can be found in the validator description in the following. The available parameters are identical to those of the validator element.

View elements are \textit{mapped} to model elements that are in turn accessed through a content provider.
\begin{lstlisting}
map|unmap
	<[ContainerElement] | [ContentElement]>
	to|from
	<[ContentProvider.Attribute]>

\end{lstlisting}

\textit{CallTasks} call a different action.
\begin{lstlisting}
call
	<[CustomAction] | [CombinedAction] | [SimpleAction]>
\end{lstlisting}

\textit{CombinedActions} allow the composition of actions.
\begin{lstlisting}
CombinedAction NAME {
	actions <Action> <...>
}
\end{lstlisting}

\textit{Validators} are used to validate user input. For each validator type corresponding parameters can be assigned. The \lstinline!message! parameter is valid for every type and will be shown to the user if the validation fails.

The RegExValidator allows the definition of a regular expression that is used to validate the user input.
\begin{lstlisting}
validator RegExValidator NAME (message STRING regEx STRING)
\end{lstlisting}

The IsIntValidator checks whether the user input is a valid integer.
\begin{lstlisting}
validator IsIntValidator NAME (message STRING)
\end{lstlisting}

The IsNumberValidator checks whether the user input is a valid integer or float value.
\begin{lstlisting}
validator IsNumberValidator NAME (message STRING)
\end{lstlisting}

The IsDateValidator allows to define a format that the date at hand shall conform to.
\begin{lstlisting}
validator IsDateValidator NAME (message STRING format STRING)
\end{lstlisting}

The NumberRangeValidator allows the definition of a numeric range that shall contain the user input.
\begin{lstlisting}
validator NumberRangeValidator NAME (message STRING min FLOAT max FLOAT )
\end{lstlisting}

The StringRangeValidator allows the definition of a string length range. The length of the \lstinline!STRING! input by the user will be checked against this range.
\begin{lstlisting}
validator StringRangeValidator NAME (message STRING minLength INT maxLength INT)
\end{lstlisting}

The NotNullValidator makes the input field required.
\begin{lstlisting}
validator NotNullValidator NAME (message STRING)
\end{lstlisting}

The \textit{RemoteValidator} allows to use a validator offered by the backend server. By default only the content and id of the field on which the RemoteValidator has been assigned are transmitted to the backend server. However, additional information can be provided using the \lstinline!provideModel! or
\lstinline!provideAttributes! keyword.
\begin{lstlisting}
validator RemoteValidator NAME (message STRING connection <RemoteConnection> model <ContentProvider>) |
validator RemoteValidator NAME (message STRING connection <RemoteConnection> attributes <ContentProvider.Attribute> <...>)
\end{lstlisting}

\paragraph{Process Chains}
\label{sec:processChain}
A ProcessChain is used to define several steps in which the workflow element can currently be. It is possible to define several process chains. Process chains can be nested and there can be only one active process chain.
\begin{lstlisting}
ProcessChain NAME {
	<ProcessChainStep> <...>
}
\end{lstlisting}

Each ProcessChainStep specifies a view that will be displayed once the process chain step becomes the current process chain step of the active process chain. Additionally, conditions can be defined that restrict switching to the next or previous process chain step. Events can trigger changing to the next or previous process chain step.
 
Instead of the default of proceeding and reversing along the process chain, another step in the process chain will become active as successor or predecessor, if referenced using \lstinline|goto| or \lstinline|returnTo|.

\begin{lstlisting}
step NAME:
	view <[ContainerElement] | [ContentElement]>
	proceed {
		on <Event>
		given <Condition>
		do <Action>
	}
	reverse on <Event>
	(goto <ProcessChainStep> | returnTo <ProcessChainStep>) on <Event>
	return on <Event>
	return and proceed on <Event>
	message STRING
\end{lstlisting}

Process chains can be refined using sub process chains.
\begin{lstlisting}
step NAME:
	subProcessChain <ProcessChain>
\end{lstlisting}

The event definition for \lstinline!EventDef! is the same as for event bindings:
\begin{lstlisting}
<Container | Content> . <elementEventType> |
<GlobalEventType> |
<OnConditionEvent> <...>
\end{lstlisting}

\paragraph{Content Providers}
Each content provider manages one instance of an entity. View fields are not mapped directly to a model element, but only content providers can be mapped to view elements. Data instances of the content providers can be updated or persisted using DataActions.

It allows to create or update (save), read (load) or delete (remove) the stored instance. Which of those operations is possible is specified in \lstinline!allowedOperations!. By default all operations are allowed. A filter enables to query a subset of all saved instances. The \lstinline!providerType! defines whether the instances shall be stored locally or remotely.

\paragraph{Remote Connections} 
A remote connection allows specifying a URI for the backend communication as well as a path to specify the storage location of files to be uploaded. The backend must comply with the \MD web service interface as specified in \Cref{chp:appendix:backend-ws-spec}.
\begin{lstlisting}
remoteConnection NAME {
	uri URI
	storagePath PATH
}
\end{lstlisting}

Furthermore, the \MD framework also provides a \textit{location provider}, \ie a virtual content provider for locations. The entity which is automatically handled by this content provider contains attributes such as latitude, longitude, street, etc. In \mapapps the location provider can be used to get coordinates from a map and resolve the corresponding address.
