% !TeX spellcheck = en_GB
% !TeX program = xelatex
% !TeX root = md2-user-handbook.tex

\subsection{Setting up your \MD Model Development Environment}

The following software is required to enable modelling of \MD models:

\begin{itemize}
\item A current Eclipse IDE with support for Java EE development (e.g. Luna)
\item Using the \href{https://eclipse.org/Xtext/download.html}{Xtext Update Site}, install a current version of the Xtext redistributable
\item From the archive that you obtained, install the \MD features
\end{itemize}

You are ready to go.


\subsection{Setting up your \mapapps Development Environment in NetBeans}
\label{subsec:basic-setup}

As a prerequisite, ensure that the following software is installed:

\begin{itemize}
\item \mapapps 3.1.0
\item NetBeans EE (e.g. Version 8.0)
\item Apache Tomcat 7.0 with running \mapapps runtime\footnote{For details on their installation, please refer to the \mapapps documentation.}
\end{itemize}

\begin{enumerate}
\item {Set up your \mapapps  development environment in NetBeans}

\begin{enumerate}
\item Extract the \lstinline|sampleProjRemote| project from the \mapapps distribution and load it in NetBeans. \label{item:extraction}
\item In its \lstinline|pom.xml|,
 set the \lstinline|mapapps.remote.base| directive to the URL where the \mapapps runtime is installed.
\item Start the Jetty web server from the context menu of the project.

\end{enumerate}

\item Deploy the generic \MD runtime bundles by copying the \MD runtime bundles into the directory \lstinline[language=Simple]|src/main/js/bundles/| in the project:
	\begin{quotation}
		 \lstinline|md2_formcontrols|
		 
		 \lstinline|md2_list_of_open_issues|
		 
		 \lstinline|md2_local_store|
		 
		 \lstinline|md2_location_service|
		 
		 \lstinline|md2_runtime|
		 		 
		 \lstinline|md2_store|
		 
		 \lstinline|md2_workflow_store|
		 
		 \lstinline|onlinestatus|
	\end{quotation}



\end{enumerate}

\subsection{Setting up GlassFish to Run Generated Backends}
To enable deployment of generated backends, follow the subsequent steps:

%Hint for developers of MD2:
%As you typically work in at least two instances of Eclipse, you are now free to choose which one you want to use for managing the backend.
%However, we suggest to use the generated instance, since this way all your app-related contents will be within this generated instance.
%All MD2 framework-related contents reside in the other instance.

\begin{enumerate}
\item In Eclipse, open the Servers tab.
\item Right-click it and choose \enquote{New} $\rightarrow$ \enquote{Server}.
\item If the GlassFish adapter is not installed yet (look for \enquote{GlassFish} in the list of types), click \enquote{Download additional server adapters} and install the entry \enquote{GlassFish Tools}.
\item From the list of types, choose \enquote{GlassFish} $\rightarrow$ \enquote{GlassFish 4.0} and click \enquote{Next}.
% (in the next line, GlassFish is written with lowercase F on purpose! Field is actually named this way.)
\item For the Glassfish Server Directory field, navigate to the \lstinline[language=Simple]|glassfish/| subdirectory in your installation. If you entered the correct path, it should output something similar to
	\begin{quotation}
	\enquote{Found GlassFish Server version 4.0.0}.
	\end{quotation}
	Otherwise, follow the assistant's hints.
\item Click \enquote{Finish}.

\end{enumerate}
