% !TeX spellcheck = en_GB
% !TeX program = xelatex
% !TeX root = md2-user-handbook.tex

\section{Resource Paths}
Format: \\
VERB -- Path -- Request body \hfill \lstinline|<Status> - <Response body>|

\subsection*{Entities}

Load\\
\lstinline|GET - /<entity.name>/?filter=<filter>| \hfill \lstinline|200 OK - List<Entity>| \\
\lstinline|GET - /<entity.name>/first?filter=<filter>| \hfill \lstinline|200 OK - Entity or 404 NOT FOUND|
\\
\\
Save\\
\lstinline|PUT - /<entity.name>/ - List<Entity>| \hfill \lstinline|200 OK - List<{ “__internalId”: <id> }>|
\\
\\
Delete\\
\lstinline|DEL - /<entity.name>/<id>| \hfill \lstinline|200 OK or 404 NOT FOUND|


\subsection*{Remote Validations}
\lstinline|GET - /md2_validator/<remoteValidator.name>/ - Entity| 

~ \hfill \lstinline|200 OK - ValidationResult object|
\\ \\
\lstinline|GET - /md2_validator/<remoteValidator.name>/?attrName1=content&attrName2=content ... &attrNameN=content| 

~ \hfill \lstinline|200 OK - ValidationResult object|
\\
\\
attrNameX is a fully qualified name, having \\
\lstinline|contentProviderName.path.to.attribute|


\subsection*{Filter Parameter}
\begin{lstlisting}[language=Simple]
not <Attribute> (equals|greater|smaller|<=|>=) (<Int>|<Float>|<String>|<InputField>)
((and|or)(not)? <Attribute> (equals|greater|smaller|<=|>=) (<Int>|<Float>|<String>|<InputField>))*
\end{lstlisting}

\subsection*{Resource for Model Version Checks}
The model version should be checked by the apps for all remote connections. Requests are only valid if the server accepts the current model version. \\
\lstinline|GET /md2_model_version/current| \hfill \lstinline|200 OK - <version>| \\
\\
\lstinline|GET /md2_model_version/is_valid?version=<version>| 

~ \hfill \lstinline!200 OK - { "isValid": (true|false) }!


\section{JSON Format Conventions}
List<Entity>:
\begin{lstlisting}[language=Javascript]
{
	"entityName": [
	{
		"attribute": <Value type see below>,
		[...]
	},
	{
		"attribut": <Value type see below>,
		[...]
	} [...]
	]
}
\end{lstlisting}
Having <Entity> = Entity without root node
\\
\\
Entity:
\begin{lstlisting}[language=Javascript]
{
	"entityName": [
		"attribute": <Value type see below>,
		[...]
	}
}
\end{lstlisting}
~
\\
Validation Result:
\begin{lstlisting}[language=Javascript]
{
	"ok": (true|false),
	"error": [
	{
		"message": "Allgemeine Fehlermeldung",
		"attributes": ["attribut1", "attribut2"]
	},
	{
		"message": "can’t be blank",
		"attributes": ["forename", "surname"]
	}
	]
	[...]
}
\end{lstlisting}
~
\\ \\
Mapping for data types (language data type -> JSON type for attribute values) \\
\lstinline|Enum -> Int (index of the currently selected value)| \\
\lstinline|Int -> Number|	
\\ \\
\lstinline|Float -> Number| \\
\lstinline|<Everything else> -> String| \\
\lstinline|Date -> String im Format yyyy-mm-ddThh:mm:ss+hh:mm| \\


\section{Examples}

GET \lstinline|/customer/first| returning one customer
\begin{lstlisting}[language=Javascript]
{
	"customer": {
		"__internalId": "0",
		"firstName": "Ulrich",
		"lastName": "M\u00c3\u00bcller",
		"membership": "1",
		"professionalCategory": "0"
	}
}
\end{lstlisting}
~
\\
GET \lstinline|/customer| returning multiple customers
\begin{lstlisting}[language=Javascript]
{
	"customer": [
	{
		"__internalId": "0",
		"firstName": "Ulrich",
		"lastName": "M\u00c3\u00bcller",
		"membership": "1",
		"professionalCategory": "0"
	},
	{
		"__internalId": "0",
		"firstName": "Hans",
		"lastName": "Dampf",
		"membership": "1",
		"professionalCategory": "0"
	}
	]
}
\end{lstlisting}
